%% Python 101 -- introduction to basic use of Python 2.7
%%
%% Tim van Werkhoven (t.i.m.vanwerkhoven@xs4all.nl), 20110321
%%
%% This file is licensed under the Creative Commons Attribution-Share Alike
%% license versions 3.0 or higher, see
%% http://creativecommons.org/licenses/by-sa/3.0/

% Copyright 2004 by Till Tantau <tantau@users.sourceforge.net>.
%
% In principle, this file can be redistributed and/or modified under
% the terms of the GNU Public License, version 2.
%
% However, this file is supposed to be a template to be modified
% for your own needs. For this reason, if you use this file as a
% template and not specifically distribute it as part of a another
% package/program, I grant the extra permission to freely copy and
% modify this file as you see fit and even to delete this copyright
% notice. 

% Generate either a full presentation, or just handouts. Handouts don't 
% include the overlay things and 'compact' a presentation on less slides.
\documentclass[xetex,10pt]{beamer}
%\documentclass[xetex,mathserif,serif,10pt, handout]{beamer}

%%% Theme %%%%%%%%%%%%%%%%%%%%%%%%%%%%%%%%%%%%%%%%%%%%%%%%%%%%%%%%%%%%%%%%%%%%

\usetheme{Madrid}

\def\imgpath{./img/}

\def\green{\color{green}}
\def\orange{\color{orange}}
\def\red{\color{red}}

\newcommand{\squash}[1]{\parbox{0pt}{#1}}

\def\spacer{\vspace*{1em}}

%%% Packages %%%%%%%%%%%%%%%%%%%%%%%%%%%%%%%%%%%%%%%%%%%%%%%%%%%%%%%%%%%%%%%%%

% Prefer British spelling & hyphenation
\usepackage[british]{babel}
\usepackage{xunicode}
\usepackage{xltxtra}
\defaultfontfeatures{Mapping=tex-text}

\usepackage{graphicx}
\usepackage{amsmath, amsthm, amssymb}
\usepackage{hyperref}

% from http://blog.miliauskas.lt/2008/09/python-syntax-highlighting-in-latex.html
% from http://licejus.lt/~gintas/files/pythonlisting.tex
%\include{pythonlisting}
\usepackage{color}
\usepackage[procnames]{listings}
\usepackage{setspace}
\usepackage{textcomp}

% from http://www.nabble.com/zorder-seems-to-cause-problems-when-embed-python-in-latex-files-td19949285.html
% from http://www.imada.sdu.dk/~ehmsen/python.sty
%\usepackage{python}

\newcommand{\pypypy}[1]{\footnote[frame]{Py3: #1}}

\usepackage{tikz}
\usetikzlibrary{arrows,positioning} 

%%% Python code highlighting %%%%%%%%%%%%%%%%%%%%%%%%%%%%%%%%%%%%%%%%%%%%%%%%

\renewcommand{\lstlistlistingname}{Python Listings}
\renewcommand{\lstlistingname}{Code Listing}

\definecolor{gray}{gray}{0.5}
\definecolor{green}{rgb}{0,0.5,0}
\definecolor{lightgreen}{rgb}{0,0.7,0}
\definecolor{purple}{rgb}{0.5,0,0.5}
\definecolor{darkred}{rgb}{0.5,0,0}

%\lstdefinelanguage{pythontim}{
%basicstyle=\ttfamily\small\setstretch{1},
%numberstyle=\footnotesize,      % the size of the fonts that are used for the line-numbers
%showspaces=false,               % show spaces adding particular underscores
%showstringspaces=false,         % underline spaces within strings
%showtabs=false,                 % show tabs within strings adding particular underscores
%stringstyle=\color{green},
%alsoletter={1234567890},
%keywordstyle=\color{blue},
%emph={access,and,as,break,class,continue,def,del,elif,else,%
%except,exec,finally,for,from,global,if,import,in,is,%
%lambda,not,or,pass,print,raise,return,try,while,assert},
%emphstyle=\color{orange}\bfseries,
%tabsize=2,
%emph={[2]self},
%emphstyle=[2]\color{gray},
%emph={[4]ArithmeticError,AssertionError,AttributeError,BaseException,%
%DeprecationWarning,EOFError,Ellipsis,EnvironmentError,Exception,%
%False,FloatingPointError,FutureWarning,GeneratorExit,IOError,%
%ImportError,ImportWarning,IndentationError,IndexError,KeyError,%
%KeyboardInterrupt,LookupError,MemoryError,NameError,None,%
%NotImplemented,NotImplementedError,OSError,OverflowError,%
%PendingDeprecationWarning,ReferenceError,RuntimeError,RuntimeWarning,%
%StandardError,StopIteration,SyntaxError,SyntaxWarning,SystemError,%
%SystemExit,TabError,True,TypeError,UnboundLocalError,UnicodeDecodeError,%
%UnicodeEncodeError,UnicodeError,UnicodeTranslateError,UnicodeWarning,%
%UserWarning,ValueError,Warning,ZeroDivisionError,abs,all,any,apply,%
%basestring,bool,buffer,callable,chr,classmethod,cmp,coerce,compile,%
%complex,copyright,credits,delattr,dict,dir,divmod,enumerate,eval,%
%execfile,exit,file,filter,float,frozenset,getattr,globals,hasattr,%
%hash,help,hex,id,input,int,intern,isinstance,issubclass,iter,len,%
%license,list,locals,long,map,max,min,object,oct,open,ord,pow,property,%
%quit,range,raw_input,reduce,reload,repr,reversed,round,set,setattr,%
%slice,sorted,staticmethod,str,sum,super,tuple,type,unichr,unicode,%
%vars,xrange,zip},
%emphstyle=[4]\color{purple}\bfseries,
%escapechar=\~,
%upquote=true,
%moredelim=**[is][\uncover<+>]{@}{@},
%morecomment=[s][\color{lightgreen}]{"""}{"""},
%commentstyle=\color{red}\slshape,
%literate={>>>}{\textbf{\textcolor{darkred}{>{>}>}}}3%
%         {...}{{\textcolor{gray}{...}}}3,
%procnamekeys={def,class},
%procnamestyle=\color{blue}\textbf,
%framexleftmargin=1mm, framextopmargin=1mm, frame=shadowbox,
%rulesepcolor=\color{blue},
%}

%\lstnewenvironment{python}[1][]{
%\lstset{
%language=pythontim,#1
%%language=Python,#1
%}}{}

%\def\pythoni{\lstinline[language=pythontim]}
%\def\pythoni{\lstinline[language=Python]}

% This gives syntax highlighting in the python environment
\renewcommand{\lstlistlistingname}{Code Listings}
\renewcommand{\lstlistingname}{Code Listing}
\definecolor{gray}{gray}{0.5}
\definecolor{key}{rgb}{0,0.5,0}

\lstdefinelanguage{pythontim}{
basicstyle=\ttfamily\small\setstretch{1},
stringstyle=\color{red},
showstringspaces=false,
alsoletter={1234567890},
%otherkeywords={\ , \}, \{},
keywordstyle=\color{blue},
emph={access,and,as,break,class,continue,def,del,elif,else,%
except,exec,finally,for,from,global,if,import,in,is,%
lambda,not,or,pass,print,raise,return,try,while,assert},
emphstyle=\color{orange}\bfseries,
emph={[2]True, False, None, self},
emphstyle=[2]\color{green},
emph={[3]from, import, as},
emphstyle=[3]\color{blue},
upquote=true,
%moredelim=**[is][\only<+>{\makebox[0cm]{•}}\hspace{1em}]{@}{@},
moredelim=**[is][\only<+>{\hspace{-2mm}\makebox[2mm]{•}}]{@}{@},
morecomment=[s]{"""}{"""},
commentstyle=\color{gray}\slshape,
%emph={[4]1, 2, 3, 4, 5, 6, 7, 8, 9, 0},
%emphstyle=[4]\color{blue},
literate=*{:}{{\textcolor{blue}:}}{1}%
	{=}{{\textcolor{blue}=}}{1}%
	{-}{{\textcolor{blue}-}}{1}%
	{+}{{\textcolor{blue}+}}{1}%
	{*}{{\textcolor{blue}*}}{1}%
	{!}{{\textcolor{blue}!}}{1}%
	{(}{{\textcolor{blue}(}}{1}%
	{)}{{\textcolor{blue})}}{1}%
	{[}{{\textcolor{blue}[}}{1}%
	{]}{{\textcolor{blue}]}}{1}%
	{<}{{\textcolor{blue}<}}{1}%
	{>}{{\textcolor{blue}>}}{1},%
emph={[5]ArithmeticError,AssertionError,AttributeError,BaseException,%
DeprecationWarning,EOFError,Ellipsis,EnvironmentError,Exception,%
False,FloatingPointError,FutureWarning,GeneratorExit,IOError,%
ImportError,ImportWarning,IndentationError,IndexError,KeyError,%
KeyboardInterrupt,LookupError,MemoryError,NameError,None,%
NotImplemented,NotImplementedError,OSError,OverflowError,%
PendingDeprecationWarning,ReferenceError,RuntimeError,RuntimeWarning,%
StandardError,StopIteration,SyntaxError,SyntaxWarning,SystemError,%
SystemExit,TabError,True,TypeError,UnboundLocalError,UnicodeDecodeError,%
UnicodeEncodeError,UnicodeError,UnicodeTranslateError,UnicodeWarning,%
UserWarning,ValueError,Warning,ZeroDivisionError,abs,all,any,apply,%
basestring,bool,buffer,callable,chr,classmethod,cmp,coerce,compile,%
complex,copyright,credits,delattr,dict,dir,divmod,enumerate,eval,%
execfile,exit,file,filter,float,frozenset,getattr,globals,hasattr,%
hash,help,hex,id,input,int,intern,isinstance,issubclass,iter,len,%
license,list,locals,long,map,max,min,object,oct,open,ord,pow,property,%
quit,range,raw_input,reduce,reload,repr,reversed,round,set,setattr,%
slice,sorted,staticmethod,str,sum,super,tuple,type,unichr,unicode,%
vars,xrange,zip},
emphstyle=[5]\color{purple}\bfseries,
escapechar=\~,
xleftmargin=2mm,
%xrightmargin=2mm,
framexleftmargin=3mm, framextopmargin=1mm, frame=shadowbox,
rulesepcolor=\color{blue},
}

\lstnewenvironment{python}[1][]{
\lstset{
language=pythontim,#1
}}{}

\def\pythoni{\lstinline[language=pythontim]}

%%% TikZ settings %%%%%%%%%%%%%%%%%%%%%%%%%%%%%%%%%%%%%%%%%%%%%%%%%%%%%%%%%%%

% For every picture that defines or uses external nodes, you'll have to
% apply the 'remember picture' style. To avoid some typing, we'll apply
% the style to all pictures.
\tikzstyle{every picture}+=[remember picture]
\tikzstyle{na} = [baseline=-.5ex]

\tikzset{
    %Define standard arrow tip
    >=stealth',
    %Define style for boxes
    punkt/.style={
           rectangle,
           rounded corners,
           draw=black, very thick,
           text width=6.5em,
           minimum height=2em,
           text centered},
    % Define arrow style
    pil/.style={
           ->,
           thick,
           shorten <=2pt,
           shorten >=2pt,}
}

%%% Presentation layout %%%%%%%%%%%%%%%%%%%%%%%%%%%%%%%%%%%%%%%%%%%%%%%%%%%%%%

% No navigation symbols please
\setbeamertemplate{navigation symbols}{}

% Cover with some transparency
%\setbeamercovered{dynamic}
\setbeamercovered{invisible}

% Delete this, if you do not want the table of contents to pop up at
% the beginning of each section:
\AtBeginSection[]
{
 \begin{frame}<beamer>
   \frametitle{Outline}
   \tableofcontents[currentsection]
 \end{frame}
}

% If you wish to uncover everything in a step-wise fashion, uncomment
% the following command: 
%\beamerdefaultoverlayspecification{<+->}

%%% Document details %%%%%%%%%%%%%%%%%%%%%%%%%%%%%%%%%%%%%%%%%%%%%%%%%%%%%%%%%

% This is only inserted into the PDF information catalog.

\hypersetup{
pdfauthor = {Tim van Werkhoven},
pdftitle = {Python 101 introduction},
pdfsubject = {Python 101 introduction},
pdfkeywords = {Python, slicing, aliases, objects, variables},
pdfcreator = {LaTeX with hyperref package},
pdfproducer = {pdfLaTeX},
colorlinks = false,
breaklinks = true,
linkcolor = red,                    % color of internal links
citecolor = green,                  % color of links to bibliography
filecolor = magenta,                % color of file links
urlcolor = blue,                    % color of external links
}

%%% Authors etc. %%%%%%%%%%%%%%%%%%%%%%%%%%%%%%%%%%%%%%%%%%%%%%%%%%%%%%%%%%%%%

\title{Python 101}
\subtitle{Introduction into Python scripting}
\author{\href{https://www.staff.science.uu.nl/~werkh108/}{Tim van Werkoven}}
\institute[SIU, UU]{%
\href{http://www.astro.uu.nl/}{Sterrekundig Instituut Utrecht}, %
\href{http://www.uu.nl}{Utrecht University}\\[1em]
\includegraphics[height=10mm]{\imgpath logo_uu.pdf}
}
\date{October 2011}

%%% Frames %%%%%%%%%%%%%%%%%%%%%%%%%%%%%%%%%%%%%%%%%%%%%%%%%%%%%%%%%%%%%%%%%%%

% Structuring a talk is a difficult task and the following structure
% may not be suitable. Here are some rules that apply for this
% solution: 

% - Exactly two or three sections (other than the summary).
% - At *most* three subsections per section.
% - Talk about 30s to 2min per frame. So there should be between about
%   15 and 30 frames, all told.

% - A conference audience is likely to know very little of what you
%   are going to talk about. So *simplify*!
% - In a 20min talk, getting the main ideas across is hard
%   enough. Leave out details, even if it means being less precise than
%   you think necessary.
% - If you omit details that are vital to the proof/implementation,
%   just say so once. Everybody will be happy with that.

\begin{document}

%\setbeameroption{show notes on second screen} 

%%%%%%%%%%%%%%%%%%%%%%%%%%%%%%%%%%%%%%%
\begin{frame}
  \titlepage
\end{frame}
%%%%%%%%%%%%%%%%%%%%%%%%%%%%%%%%%%%%%%%

%%%%%%%%%%%%%%%%%%%%%%%%%%%%%%%%%%%%%%%
\begin{frame}
    \frametitle{Outline}
    %\note{Some notes for the current slide}
    \setcounter{tocdepth}{1}
    \tableofcontents
    % You might wish to add the option [pausesections]
\end{frame}
%%%%%%%%%%%%%%%%%%%%%%%%%%%%%%%%%%%%%%%

%%%%%%%%%%%%%%%%%%%%%%%%%%%%%%%%%%%%%%%%%%%%%%%%%%%%%%%%%%%%%%%%%%%%%%%%%%%%%%
\section{Introduction}
%%%%%%%%%%%%%%%%%%%%%%%%%%%%%%%%%%%%%%%%%%%%%%%%%%%%%%%%%%%%%%%%%%%%%%%%%%%%%%

%%%%%%%%%%%%%%%%%%%%%%%%%%%%%%%%%%%%%%%
\begin{frame}
	\frametitle{Antigravity}
	\begin{center}
	\vfill
	\setbeamercovered{transparent}
	\only<1>{\Huge{Why Python?}}
	\only<2>{\begin{figure}[!h]
	\includegraphics[height=0.75\textheight]{\imgpath python.png}
	\caption{\url{http://xkcd.com/353/}}
	\end{figure}}
	\vfill
	\end{center}
\end{frame}
%%%%%%%%%%%%%%%%%%%%%%%%%%%%%%%%%%%%%%%


%%%%%%%%%%%%%%%%%%%%%%%%%%%%%%%%%%%%%%%
\begin{frame}
	\frametitle{About Python}
	\includegraphics[width=0.3\textwidth]{\imgpath python-logo-generic.pdf}
	
	\begin{itemize}
		\item High-level ({\green +IDL}, {\green +Perl})
		\item Cross-platform \& widely used ({\red -IDL}, {\green +Perl})
		\item Flexible ({\red -IDL}, {\orange $\sim$Perl})
		\item Active community ({\green +IDL}, {\green +Perl})
		\item Well-documented ({\green + IDL}, {\green +Perl})
		\item Free (GPL) ({\red -IDL}, {\green +Perl})
		\item \emph{Dutch} ({\red -IDL}, {\red -Perl})
	\end{itemize}
\end{frame}
%%%%%%%%%%%%%%%%%%%%%%%%%%%%%%%%%%%%%%%

%%%%%%%%%%%%%%%%%%%%%%%%%%%%%%%%%%%%%%%
\begin{frame}
	\frametitle{Versions}
	
	There are two branches:
	
	\begin{itemize}
	\item Version 2.x (newest: 2.7.2, June 11, 2011)
	\item Version 3.x (newest: 3.2.2, September 4, 2011)
	\end{itemize}
	
	3.x is backward \emph{in}compatible with 2.x on purpose.\\
	2.7 is the final version in the 2.x branch.\\[2em]
	
	I discuss version 2.7 (most third-party libraries available), some differences in Python 3.
	
\end{frame}
%%%%%%%%%%%%%%%%%%%%%%%%%%%%%%%%%%%%%%%

%%%%%%%%%%%%%%%%%%%%%%%%%%%%%%%%%%%%%%%
\begin{frame}[fragile]
	\frametitle{This presentation}
	
	Python code is shown like this:
\begin{python}
alist = [1,2,3,4]
for i in alist:
    print i
\end{python}
	
	\spacer
	
	Python commands and output is shown like this:

\begin{python}
>>> print "Hello world"
Hello world!
\end{python}

\end{frame}
%%%%%%%%%%%%%%%%%%%%%%%%%%%%%%%%%%%%%%%


%%%%%%%%%%%%%%%%%%%%%%%%%%%%%%%%%%%%%%%%%%%%%%%%%%%%%%%%%%%%%%%%%%%%%%%%%%%%%%
\section{Basics}
%%%%%%%%%%%%%%%%%%%%%%%%%%%%%%%%%%%%%%%%%%%%%%%%%%%%%%%%%%%%%%%%%%%%%%%%%%%%%%

\subsection{Running Python}
%%%%%%%%%%%%%%%%%%%%%%%%%%%%%%%%%%%%%%%

%%%%%%%%%%%%%%%%%%%%%%%%%%%%%%%%%%%%%%%
\begin{frame}[fragile]{Getting Python}
	
	Python can be downloaded from \url{http://python.org/download/}, but it is probably better to get is as part of a distribution.

	\spacer

	\begin{itemize}
		\item Enthought\footnote[frame]{See \url{http://enthought.com/products/epd.php}} provides a complete Python package for scientific computing
		\item Many package managers such as \verb!apt!, \verb!emerge!, \verb!port! and \verb!fink! support python
	\end{itemize}
	
\end{frame}
%%%%%%%%%%%%%%%%%%%%%%%%%%%%%%%%%%%%%%%

%%%%%%%%%%%%%%%%%%%%%%%%%%%%%%%%%%%%%%%
\begin{frame}[fragile]{Python interpreter}
	
	Python can be run in interactive mode or on files (like IDL and perl).
	
\begin{verbatim}
tim@Saturn ~ » python2.7
Python 2.7.2 (default, Aug 29 2011, 10:42:48) 
[GCC 4.2.1 (Based on Apple Inc. build 5658) (LLVM build 2335.15.00)] on darwin
Type "help", "copyright", "credits" or "license" for more information.
>>> 
\end{verbatim}

Enter commands to do something

\begin{python}
>>> print "Hello world"
Hello world!
>>> 2 + 3
5
>>> 1.0/7
0.14285714285714285
\end{python}

\end{frame}
%%%%%%%%%%%%%%%%%%%%%%%%%%%%%%%%%%%%%%%

%%%%%%%%%%%%%%%%%%%%%%%%%%%%%%%%%%%%%%%
\begin{frame}[fragile]
	\frametitle{IPython}
	
	Besides the standard python interpreter, IPython is a strongly improved version.

	\spacer

	\begin{itemize}
		\item Syntax highlighting
		\item Tab-completion
		\item Inline shell commands
		\item Tracebacks on errors
		\item Object introspection
		\item Built-in debugging through \verb!pdb!
	\end{itemize}
	
	\spacer
	IPython is available at \url{http://ipython.org/}.
	
\end{frame}
%%%%%%%%%%%%%%%%%%%%%%%%%%%%%%%%%%%%%%%

%%%%%%%%%%%%%%%%%%%%%%%%%%%%%%%%%%%%%%%
\begin{frame}[fragile]{Python scripts}
	
	Python can also be used as scripts, where the first line has to read something like
	
\begin{verbatim}
#! /usr/bin/env python
\end{verbatim}
	
	to tell the shell what interpreter to use for this file. Don't forget to \verb!chmod +x! afterwards.\\

	\pause
	\spacer

	You can use the line

\begin{verbatim}
# -*- coding: encoding -*-
\end{verbatim}

	to tell python what encoding you are using, in order to include Unicode literals, for example.

\end{frame}
%%%%%%%%%%%%%%%%%%%%%%%%%%%%%%%%%%%%%%%


\subsection{Objects}
%%%%%%%%%%%%%%%%%%%%%%%%%%%%%%%%%%%%%%%

%%%%%%%%%%%%%%%%%%%%%%%%%%%%%%%%%%%%%%%
\begin{frame}[fragile]
	\frametitle{Objects\footnote[frame]{See \href{http://docs.python.org/reference/datamodel.html\#objects-values-and-types}{http://docs.python.org/reference/datamodel.html¶3.1}}}

	\begin{itemize}
		\item Everything is stored in an object, \emph{even functions}.
		\pause
		\item Objects have identity, type and value.
		\pause
		\item Immutable (values cannot change) vs.\ mutable objects (can change)
		\pause
		%\item Objects are never destroyed, but are garbage collected
		%\pause
		\item Some objects contain other objects: \emph{containers}
	\end{itemize}
	
	\spacer
	
	\begin{python}
	>>> a = 10
	>>> (id(a), type(a), a)
	(4331745360, int, 10)
	\end{python}

\end{frame}
%%%%%%%%%%%%%%%%%%%%%%%%%%%%%%%%%%%%%%%

%%%%%%%%%%%%%%%%%%%%%%%%%%%%%%%%%%%%%%%
\begin{frame}[fragile]{Objects examples}
	
	Objects \& values

	\spacer

	\begin{columns}[T]
		\begin{column}{0.45 \textwidth}
			\begin{python}
			@>>> code = 'Python'@
			@>>> version = 2.7@
			@>>> p = code@
			@>>> code = 'Fortran'@
			\end{python}
%			\begin{python}
%			%\uncover<1->{>>> code = 'Python'}%
%			%\uncover<2->{>>> version = 2.7}%
%			%\uncover<3->{>>> p = code}%
%			%\uncover<4->{>>> code = 'Fortran'}%
%			\end{python}
		\end{column}
		\begin{column}{0.45 \textwidth}
			\begin{tabular}{l|l}
			Object & Value \\
			\hline
			\verb!code! \tikz[na] \node[coordinate] (v1) {}; & \tikz[na] \node[coordinate] (v1v) {}; `Python' \\
			\verb!version! \tikz[na] \node[coordinate] (v2) {}; & \tikz[na] \node[coordinate] (v2v) {}; 2.7 \\
			\verb!p! \tikz[na] \node[coordinate] (v3) {}; &  \\
			& \tikz[na] \node[coordinate] (v3v) {}; `Fortran' \\
			\end{tabular}
		\end{column}
	\end{columns}

	
	\begin{tikzpicture}[overlay]
		\path[->]<1-3> (v1) edge [bend left] (v1v);
		\path[->]<2-> (v2) edge [bend left] (v2v);
		\path[->]<3-3> (v3) edge [bend right] (v1v);
		\path[->]<4-> (v1) edge [bend right] (v3v);
		\path[->]<4-> (v3) edge [] (v3v);
	\end{tikzpicture}

\end{frame}
%%%%%%%%%%%%%%%%%%%%%%%%%%%%%%%%%%%%%%%

%%%%%%%%%%%%%%%%%%%%%%%%%%%%%%%%%%%%%%%
\begin{frame}[fragile]
	\frametitle{Objects examples II}

\begin{python}
>>> a = 10; b = a
>>> a is b
True
\end{python}

\pythoni{a} and \pythoni{b} are the same \emph{object}.
\pause

\begin{python}
>>> b = 20; a is b
False
>>> b = 10; a is b
True
\end{python}

\pythoni{a} and \pythoni{b} are again the same object, \emph{even though we did not assign them manually}
\pause

\begin{python}
>>> id(a), id(b)
(4331745360, 4331745360)
\end{python}

\pythoni{id()}\footnote[frame]{\url{http://docs.python.org/library/functions.html\#id}} gives the identity of an object (memory location).

\end{frame}
%%%%%%%%%%%%%%%%%%%%%%%%%%%%%%%%%%%%%%%

%%%%%%%%%%%%%%%%%%%%%%%%%%%%%%%%%%%%%%%
\begin{frame}[fragile]
	\frametitle{Objects examples III}

Two lists
\begin{python}
>>> a = [1,2,3]
>>> b = [1,2,3]
>>> a == b
True
\end{python}

\pythoni{a} and \pythoni{b} have the same \emph{value}.	
\pause

\begin{python}
>>> a is b
False
\end{python}

\pythoni{a} and \pythoni{b} are \emph{not} the same \emph{object}.	
\pause

\begin{python}
>>> a = b
>>> a is b
True
\end{python}

\pythoni{a} now points to \pythoni{b}.
\pause

\end{frame}
%%%%%%%%%%%%%%%%%%%%%%%%%%%%%%%%%%%%%%%


%%%%%%%%%%%%%%%%%%%%%%%%%%%%%%%%%%%%%%%
\begin{frame}[fragile]
	\frametitle{Objects examples IV}

\begin{python}
>>> a[0] = 10
>>> a is b
True
>>> a
[10, 2, 3]
>>> b
[10, 2, 3]
\end{python}

Since \pythoni{a} points to \pythoni{b}, if we change \pythoni{a}, we also change \pythoni{b}.

	\spacer

\emph{(This subtle difference arises from the fact that a list is mutable.)}

\end{frame}
%%%%%%%%%%%%%%%%%%%%%%%%%%%%%%%%%%%%%%%

\subsection{Datatypes}
%%%%%%%%%%%%%%%%%%%%%%%%%%%%%%%%%%%%%%%

%%%%%%%%%%%%%%%%%%%%%%%%%%%%%%%%%%%%%%%
\begin{frame}[fragile]
	\frametitle{Datatypes}
	
	The different datatypes in Python include:
	\spacer
	\begin{itemize}
		\item Numbers (\pythoni{int}, \pythoni{long}, \pythoni{bool}, \pythoni{float}, \pythoni{complex})
		\item Sequences (strings,  unicode, tuples, lists)
		\item Mappings (dictionaries)
		\item Callable (functions, methods, generators)
		\item \ldots (modules, classes, files)
	\end{itemize}
	
	\spacer
	See \url{http://docs.python.org/reference/datamodel.html\#3.2} for more details.
\end{frame}
%%%%%%%%%%%%%%%%%%%%%%%%%%%%%%%%%%%%%%%

%%%%%%%%%%%%%%%%%%%%%%%%%%%%%%%%%%%%%%%
\begin{frame}[fragile]
	\frametitle{Datatypes -- Numbers}

	\begin{itemize}
		\item int: 32 bit signed number
		\item long\pypypy{all ints are long}: arbitrary length numer.
		\item float: machine-level double precision floating point numbers
		\item complex: pair of floats
		\item bool: truth values \pythoni{True} and \pythoni{False}
	\end{itemize}
\end{frame}
%%%%%%%%%%%%%%%%%%%%%%%%%%%%%%%%%%%%%%%

%%%%%%%%%%%%%%%%%%%%%%%%%%%%%%%%%%%%%%%
\begin{frame}[fragile]
	\frametitle{Datatypes -- Sequences \& mapping}

	\begin{itemize}
		\item string \& unicode\pypypy{all strings are unicode}: array of characters
		\item tuples: set of objects, immutable
		\item list: set of objects, mutable
		\item dictionary: set of objects index by arbitrary values
	\end{itemize}
\end{frame}
%%%%%%%%%%%%%%%%%%%%%%%%%%%%%%%%%%%%%%%


%%%%%%%%%%%%%%%%%%%%%%%%%%%%%%%%%%%%%%%
\begin{frame}[fragile]
	\frametitle{Datatypes -- Examples}

	Integers, normal (32 and 64 bit) and `bignum' (arbitrary length)\pypypy{all ints are of `bignum' type}:

	\spacer

\begin{python}
>>> 2**16
65536
>>> 2**512
1340780792994259709957402499820584612747936582059239337772356
1443721764030073546976801874298166903427690031858186486050853
753882811946569946433649006084096L
>>> math.sin(2**512)
0.9905786583658018
\end{python}

	\spacer

Note that bignum can generally not be used in external libraries.
\end{frame}
%%%%%%%%%%%%%%%%%%%%%%%%%%%%%%%%%%%%%%%


%%%%%%%%%%%%%%%%%%%%%%%%%%%%%%%%%%%%%%%
\begin{frame}[fragile]
	\frametitle{Datatypes -- Examples}

Floats\pypypy{\pythoni{int} divisions work differently}:
\begin{python}
>>> 5.0 / 3
1.6666666666666667
>>> 5 / 3
1
>>> 5.0 // 3
1.0
\end{python}
\end{frame}
%%%%%%%%%%%%%%%%%%%%%%%%%%%%%%%%%%%%%%%


%%%%%%%%%%%%%%%%%%%%%%%%%%%%%%%%%%%%%%%
\begin{frame}[fragile]
	\frametitle{Datatypes -- Examples}

Complex:
\begin{python}
>>> 1+2j + 1-1j
(2+1j)
>>> a=0j
>>> type(a)
complex
\end{python}

	\spacer

Boolean
\begin{python}
>>> 5 > 4
True
>>> 1 and True
True
\end{python}

\end{frame}
%%%%%%%%%%%%%%%%%%%%%%%%%%%%%%%%%%%%%%%


%%%%%%%%%%%%%%%%%%%%%%%%%%%%%%%%%%%%%%%
\begin{frame}[fragile]
	\frametitle{Datatypes -- Examples}

String
\begin{python}
>>> "Hello world"
'Hello World'
>>> "Hello" * 5
'HelloHelloHelloHelloHello'
\end{python}

	\spacer
	
Strings are actually lists
\begin{python}
>>> msg = "Hello world"
>>> msg[0]
'H'
>>> msg[0:5]
'Hello'
\end{python}
\end{frame}
%%%%%%%%%%%%%%%%%%%%%%%%%%%%%%%%%%%%%%%

\subsection{Containers}
%%%%%%%%%%%%%%%%%%%%%%%%%%%%%%%%%%%%%%%

%%%%%%%%%%%%%%%%%%%%%%%%%%%%%%%%%%%%%%%
\begin{frame}[fragile]
	\frametitle{Containers}

	Containers: tuple, list, dict

	\spacer

	\begin{itemize}
	\item Tuple: holds arbitrary objects, immutable
		\begin{itemize}
		\item \emph{Usage: coordinates, shapes, properties}
		\end{itemize}
	\item List: holds arbitrary objects, mutable
		\begin{itemize}
		\item \emph{Usage: sequences of data}
		\end{itemize}
	\item Dict: holds key$\rightarrow$value mappings, mutable
		\begin{itemize}
		\item \emph{Usage: arbitrary properties}
		\end{itemize}
	\end{itemize}
	
\end{frame}
%%%%%%%%%%%%%%%%%%%%%%%%%%%%%%%%%%%%%%%

%%%%%%%%%%%%%%%%%%%%%%%%%%%%%%%%%%%%%%%
\begin{frame}[fragile]
	\frametitle{Containers -- Tuples}

Tuples
\begin{python}
>>> a = (1, 2, 3)
>>> a
(1, 2, 3)
>>> a[1]
1
\end{python}

	\spacer


Tuples are \emph{immutable}:
\begin{python}
>>> a[1] = 2
TypeError: 'tuple' object does not support item assignment
\end{python}
\end{frame}
%%%%%%%%%%%%%%%%%%%%%%%%%%%%%%%%%%%%%%%

%%%%%%%%%%%%%%%%%%%%%%%%%%%%%%%%%%%%%%%
\begin{frame}[fragile]
	\frametitle{Containers -- Lists}

Lists are mutable, ordered sequences and can hold different datatypes.
\begin{python}
>>> a = [1, 2, 3]
>>> a.append("Hello")
>>> a
[1, 2, 3, 'Hello']
>>> a[1]
2
>>> a.pop()
'Hello'
>>> a
[1, 2, 3]
\end{python}

\end{frame}
%%%%%%%%%%%%%%%%%%%%%%%%%%%%%%%%%%%%%%%

%%%%%%%%%%%%%%%%%%%%%%%%%%%%%%%%%%%%%%%
\begin{frame}[fragile]
	\frametitle{Containers -- Dictionaries}

Dictionaries are mutable, unordered sequences and hold values by a certain key.
\begin{python}
>>> a = {'length': 3, 'width': 8, 5.0: 'five point zero'}
>>> a
{5.0: 'five point zero', 'length': 3, 'width': 8}
>>> a['length'] += 0.5
>>> a
{5.0: 'five point zero', 'length': 3.5, 'width': 8}
>>> a[1]
KeyError: 0
\end{python}


	\spacer

You can request the keys and values separately:

\begin{python}
>>> a.keys()
['width', 'length', 5.0]
>>> a.values()
[8, 4, 'five point zero']
\end{python}

\end{frame}
%%%%%%%%%%%%%%%%%%%%%%%%%%%%%%%%%%%%%%%

\subsection{Tests}
%%%%%%%%%%%%%%%%%%%%%%%%%%%%%%%%%%%%%%%

%%%%%%%%%%%%%%%%%%%%%%%%%%%%%%%%%%%%%%%
\begin{frame}[fragile]
	\frametitle{Tests}

	\begin{itemize}
	\item Use \pythoni{if}, \pythoni{elif}, \pythoni{else}
	\item Python uses \emph{whitespace} for scope
	\item Whitespace can be spaces or tabs (4 spaces recommended\footnote[frame]{See PEP 8 \url{http://www.python.org/dev/peps/pep-0008/}})
	\end{itemize}

\begin{python}
if x > 0:
    print "Positive!"
elif x < 0:
    print "Negative!"
else:
    print "Zero!"
\end{python}

	\spacer

\begin{python}
if ((x > 0) and (y > 0)) or (z < 0):
    print "Amazing!"
\end{python}

\end{frame}
%%%%%%%%%%%%%%%%%%%%%%%%%%%%%%%%%%%%%%%

\subsection{Loops}
%%%%%%%%%%%%%%%%%%%%%%%%%%%%%%%%%%%%%%%

%%%%%%%%%%%%%%%%%%%%%%%%%%%%%%%%%%%%%%%
\begin{frame}[fragile]
	\frametitle{Loops}
	Python knows \pythoni{for} \& \pythoni{while}:

\begin{python}
alist = [1,2,3,4,5,6,7,8,9,10]
for i in alist:
    print i
\end{python}

	\spacer


\begin{python}
x = 0
while x < 10:
    x = x + 1
    print x
\end{python}

	\spacer


Both snippets above print 1 through 10 inclusive on separate lines.

\end{frame}
%%%%%%%%%%%%%%%%%%%%%%%%%%%%%%%%%%%%%%%

\subsection{Functions}
%%%%%%%%%%%%%%%%%%%%%%%%%%%%%%%%%%%%%%%

%%%%%%%%%%%%%%%%%%%%%%%%%%%%%%%%%%%%%%%
\begin{frame}[fragile]
	\frametitle{Functions\footnote[frame]{More information in Software Carpentries `Python -- Functions'}}

	Use \pythoni{def} for functions

\begin{python}
def hello():
    print "Hello!"
\end{python}
\begin{python}
>>> hello()
Hello!
\end{python}

	\spacer

Functions are objects
\begin{python}
>>> (id(hello), type(hello), hello)
(4449393912, <type 'function'>, <function hello at 0x109345cf8>)
>>> func_holder = hello
>>> func_holder()
Hello!
\end{python}
\end{frame}
%%%%%%%%%%%%%%%%%%%%%%%%%%%%%%%%%%%%%%%

%%%%%%%%%%%%%%%%%%%%%%%%%%%%%%%%%%%%%%%
\begin{frame}[fragile]
	\frametitle{Functions}

	Arguments can be mandatory or optional.
	
\begin{python}
def hello2(what):			# what is mandatory
    print "Hello " + what + "!"

def hello3(what="World"):	# what is optional, default given
    print "Hello " + what + "!"
\end{python}

	\spacer

\begin{python}
>>> hello2('class')
Hello class!
>>> hello3('class')
Hello class!
>>> hello3()
Hello World!
\end{python}
\end{frame}
%%%%%%%%%%%%%%%%%%%%%%%%%%%%%%%%%%%%%%%

%%%%%%%%%%%%%%%%%%%%%%%%%%%%%%%%%%%%%%%
\begin{frame}[fragile]
	\frametitle{Functions}
	Return type is dynamic

\begin{python}
def sum_two(x, y):
    return x + y
\end{python}

	\spacer

\begin{python}
>>> sum_two(1, 1.5)
2.5
>>> sum_two("Hello", "World")
'HelloWorld'
>>> sum_two([1,2,3], [4, 5])
[1, 2, 3, 4, 5]
\end{python}

\end{frame}
%%%%%%%%%%%%%%%%%%%%%%%%%%%%%%%%%%%%%%%

%%%%%%%%%%%%%%%%%%%%%%%%%%%%%%%%%%%%%%%%%%%%%%%%%%%%%%%%%%%%%%%%%%%%%%%%%%%%%%
\section{More advanced}
%%%%%%%%%%%%%%%%%%%%%%%%%%%%%%%%%%%%%%%%%%%%%%%%%%%%%%%%%%%%%%%%%%%%%%%%%%%%%%

\subsection{Slicing}
%%%%%%%%%%%%%%%%%%%%%%%%%%%%%%%%%%%%%%%

%%%%%%%%%%%%%%%%%%%%%%%%%%%%%%%%%%%%%%%
\begin{frame}[fragile]
	\frametitle{Slicing\footnote[frame]{More information in Software Carpentries `Python -- Slicing'}}
\vfill

	\emph{Slicing} is advanced list indexing. In general: \pythoni{list[start:end:step]}
\vfill

\begin{python}
>>> alist = range(15)
>>> alist
[0, 1, 2, 3, 4, 5, 6, 7, 8, 9, 10, 11, 12, 13, 14]
\end{python}

\vfill

\begin{tabular}{ll@{}l@{}l@{}l@{}lll@{}l@{}l}
index & 0 & 1 & 2 & 3 & 4 &  & 12 & 13 & 14 \\
value & %
\fbox{\squash{0}\phantom{xx}} & %
\fbox{\squash{1}\phantom{xx}} & %
\fbox{\squash{3}\phantom{xx}} & %
\fbox{\squash{4}\phantom{xx}} & %
\fbox{\squash{5}\phantom{xx}} & %
\ldots & %
\fbox{\squash{12}\phantom{xx}} & %
\fbox{\squash{13}\phantom{xx}} & %
\fbox{\squash{14}\phantom{xx}}\\
index & -15 & -14 & -13 & -12 & -11 &  & -3 & -2 & -1 \\

\end{tabular}

\vfill

\begin{python}
>>> alist[0]
0
>>> alist[-15]
0
\end{python}

\end{frame}
%%%%%%%%%%%%%%%%%%%%%%%%%%%%%%%%%%%%%%%

%%%%%%%%%%%%%%%%%%%%%%%%%%%%%%%%%%%%%%%
\begin{frame}[fragile]
	\frametitle{Slicing}

Start at element 1, stop at (before) element 4

\begin{python}
>>> alist[1:4]
[1, 2, 3]
\end{python}

Stop at (before) element -5

\begin{python}
>>> alist[:-5]
[0, 1, 2, 3, 4, 5, 6, 7, 8, 9]
\end{python}

Show every second element

\begin{python}
>>> alist[::2]
[0, 2, 4, 6, 8, 10, 12, 14]
\end{python}

Start at element 10, step back, stop at (before) element 5 

\begin{python}
>>> alist[10:5:-1]
[10, 9, 8, 7, 6]
\end{python}

\end{frame}
%%%%%%%%%%%%%%%%%%%%%%%%%%%%%%%%%%%%%%%

%%%%%%%%%%%%%%%%%%%%%%%%%%%%%%%%%%%%%%%
\begin{frame}[fragile]
	\frametitle{Slicing}

Python checks bounds when indexing, but truncates when slicing:

\begin{python}
>>> alist[500]
IndexError: list index out of range
>>> alist[10:500]
[10, 11, 12, 13, 14]
\end{python}

	\spacer

\pythoni{alist[x:y]} is empty when \pythoni{x >= y}.

\begin{python}
>>> alist[1:1]
[]
>>> alist[2:1]
[]
>>> alist[500:1]
[]
\end{python}

\end{frame}
%%%%%%%%%%%%%%%%%%%%%%%%%%%%%%%%%%%%%%%

\subsection{Aliases}
%%%%%%%%%%%%%%%%%%%%%%%%%%%%%%%%%%%%%%%

%%%%%%%%%%%%%%%%%%%%%%%%%%%%%%%%%%%%%%%
\begin{frame}[fragile]
	\frametitle{Aliases}

	Different \emph{aliases} point to the same data.
	\spacer

	This does not matter when variables are immutable:
	\spacer

	\begin{columns}[T]
		\begin{column}{0.45 \textwidth}
			\begin{python}
			@>>> foo = 'hello'@
			@>>> bar = foo@
			@>>> bar = bar + ' world!'@
			\end{python}
		\end{column}
		\begin{column}{0.45 \textwidth}
			\begin{tabular}{l|l}
			Variable & Value \\
			\hline
			\verb!foo! \tikz[na] \node[coordinate] (v1) {}; & \tikz[na] \node[coordinate] (v1v) {};`hello' \\
			\verb!bar! \tikz[na] \node[coordinate] (v2) {}; & \tikz[na] \node[coordinate] (v2v) {};`hello world!' \\
			\end{tabular}
		\end{column}
	\end{columns}
	
	\begin{tikzpicture}[overlay]
		\path[->]<1-> (v1) edge [bend left] (v1v);
		\path[->]<2-> (v2) edge [bend right] (v1v);
		\path[->]<3-> (v2) edge [bend right] (v2v);
	\end{tikzpicture}

\end{frame}
%%%%%%%%%%%%%%%%%%%%%%%%%%%%%%%%%%%%%%%

%%%%%%%%%%%%%%%%%%%%%%%%%%%%%%%%%%%%%%%
\begin{frame}[fragile]
	\frametitle{Aliases}

	When variables are mutable, this can give unexpected results.
	\spacer
	
	\begin{columns}[T]
		\begin{column}{0.45 \textwidth}
			\begin{python}
			@>>> alist = [1, 2, 3, 4]@
			@>>> blist = alist@
			@>>> blist[0] = 5@
			\end{python}
		\end{column}
		\begin{column}{0.45 \textwidth}
			\begin{tabular}{l|l}
			Variable & Value \\
			\hline
			\verb!alist! \tikz[na] \node[coordinate] (v1) {}; & \tikz[na] \node[coordinate] (v1v) {};\verb![1, 2, 3, 4]! \\
			\verb!blist! \tikz[na] \node[coordinate] (v2) {}; & \tikz[na] \node[coordinate] (v2v) {};\verb![5, 2, 3, 4]! \\
			\end{tabular}
		\end{column}
	\end{columns}
	
	\begin{tikzpicture}[overlay]
		\path[->]<1-2> (v1) edge [bend left] (v1v);
		\path[->]<2> (v2) edge [bend right] (v1v);
		\path[->]<3-> (v1) edge [bend left] (v2v);
		\path[->]<3-> (v2) edge [bend right] (v2v);
	\end{tikzpicture}

\end{frame}
%%%%%%%%%%%%%%%%%%%%%%%%%%%%%%%%%%%%%%%

%%%%%%%%%%%%%%%%%%%%%%%%%%%%%%%%%%%%%%%
\begin{frame}[fragile]
	\frametitle{Aliases -- Functions}

	Variables are \emph{passed by object} to functions.
	\spacer

	If the variable is not mutable, it acts like `pass by value':
	\spacer

	\begin{columns}[T]
		\begin{column}{0.45 \textwidth}
			\begin{python}
			@@>>> a = 1
			@@>>> sum_two(a, 4)
			@@>>> a
			1
			\end{python}
		\end{column}
		\begin{column}{0.45 \textwidth}
			\begin{tabular}{ll}
			Object & Value \\
			\hline
			\verb!a! \tikz[na] \node[coordinate] (v1) {}; & \tikz[na] \node[coordinate] (v1v) {}; 1 \\
			\verb!x! \tikz[na] \node[coordinate] (v2) {}; & \tikz[na] \node[coordinate] (v2v) {}; 5 \\
			\verb!y! \tikz[na] \node[coordinate] (v3) {}; & \tikz[na] \node[coordinate] (v3v) {}; 4 \\
			\end{tabular}
		\end{column}
	\end{columns}

	
	\begin{tikzpicture}[overlay]
		\path[->]<1-> (v1) edge [bend left] (v1v);
		\path[->]<2> (v2) edge [bend left] (v1v);
		\path[->]<2-> (v3) edge [bend left] (v3v);
		\path[->]<3> (v2) edge [bend left] (v2v);
	\end{tikzpicture}


\end{frame}
%%%%%%%%%%%%%%%%%%%%%%%%%%%%%%%%%%%%%%%

%%%%%%%%%%%%%%%%%%%%%%%%%%%%%%%%%%%%%%%
\begin{frame}[fragile]
	\frametitle{Aliases -- Functions}

	If the variable is mutable, it is `passed by reference':
	\spacer

\begin{python}
def inc_list(l1):
    for i in range(len(l1)):
        l1[i] += 1
    return l1
\end{python}

	\spacer

	\begin{columns}[T]
		\begin{column}{0.45 \textwidth}
			\begin{python}
			@@>>> a = [1,2,3]
			@@>>> inc_list(a)
			@@>>> a
			[2,3,4]
			\end{python}
		\end{column}
		\begin{column}{0.45 \textwidth}
			\begin{tabular}{lll}
			Object & Object & Value \\
			\hline
			\verb!a!\tikz[na] \node[coordinate] (v1) {}; &%
				\fbox{\phantom{x}}\tikz[na] \node[coordinate] (v1v1) {}; &%
				\tikz[na] \node[coordinate] (v1vv1) {}; 1 \\
			 & 
				\tikz[na] \node[coordinate] (v1v) {};%
				\fbox{\phantom{x}}\tikz[na] \node[coordinate] (v1v2) {}; & %
				\tikz[na] \node[coordinate] (v1vv2) {}; 2 \\
			 & 
				\fbox{\phantom{x}}\tikz[na] \node[coordinate] (v1v3) {}; & %
				\tikz[na] \node[coordinate] (v1vv3) {}; 3 \\
			 &  & %
				\tikz[na] \node[coordinate] (v1vv4) {}; 4 \\
			\end{tabular}
		\end{column}
	\end{columns}
	
	\begin{tikzpicture}[overlay]
		\path[->]<1-> (v1) edge [bend left] (v1v);
		\path[->]<1> (v1v1) edge [bend right] (v1vv1);
		\path[->]<1> (v1v2) edge [bend right] (v1vv2);
		\path[->]<1> (v1v3) edge [bend right] (v1vv3);
		\path[->]<3-> (v1v1) edge [bend right] (v1vv2);
		\path[->]<3-> (v1v2) edge [bend right] (v1vv3);
		\path[->]<3-> (v1v3) edge [bend right] (v1vv4);
	\end{tikzpicture}


\end{frame}
%%%%%%%%%%%%%%%%%%%%%%%%%%%%%%%%%%%%%%%

\subsection{Functional programming}
%%%%%%%%%%%%%%%%%%%%%%%%%%%%%%%%%%%%%%%

%%%%%%%%%%%%%%%%%%%%%%%%%%%%%%%%%%%%%%%
\begin{frame}[fragile]{Functional programming -- filter}
	
	Python has three built-in functions that are very useful for lists, \pythoni{filter}, \pythoni{map} and \pythoni{reduce}.
	

	\spacer
	
	\pythoni{filter(func, sequence)} returns only items from \pythoni{sequence} for which \pythoni{func(item)} is true:
	
	\begin{python}
	>>> def f(x): return x % 2 != 0 and x % 3 != 0
	...
	>>> filter(f, range(2, 25))
	[5, 7, 11, 13, 17, 19, 23]
	\end{python}
\end{frame}
%%%%%%%%%%%%%%%%%%%%%%%%%%%%%%%%%%%%%%%

%%%%%%%%%%%%%%%%%%%%%%%%%%%%%%%%%%%%%%%
\begin{frame}[fragile]{Functional programming -- map}

	\pythoni{map(func, sequence)} calls \pythoni{func(item)} on all items of \pythoni{sequence} and returns a list of return values:
	
	\begin{python}
>>> def cube(x): return x*x*x
...
>>> map(cube, range(1, 11))
[1, 8, 27, 64, 125, 216, 343, 512, 729, 1000]
	\end{python}
\end{frame}
%%%%%%%%%%%%%%%%%%%%%%%%%%%%%%%%%%%%%%%

%%%%%%%%%%%%%%%%%%%%%%%%%%%%%%%%%%%%%%%
\begin{frame}[fragile]{Functional programming -- reduce}
	\pythoni{reduce(func, sequence)} returns a single value constructed by calling the binary function \pythoni{func} on the first two items of the sequence, then on the result and the next item, and so on.
	
\begin{python}
>>> def add(x,y): return x+y
...
>>> reduce(add, range(1, 5))
10
>>> add(add(add(1,2), 3), 4)
10
\end{python}

\end{frame}
%%%%%%%%%%%%%%%%%%%%%%%%%%%%%%%%%%%%%%%

%%%%%%%%%%%%%%%%%%%%%%%%%%%%%%%%%%%%%%%
\begin{frame}[fragile]{List comprehension}

\emph{List comprehensions} provide a concise way to create lists based on existing lists.
	
\begin{python}
>>> vec = [2, 4, 6]
>>> [3*x for x in vec]
[6, 12, 18]
>>> [3*x for x in vec if x > 3]
[12, 18]
\end{python}

Generate tuples as list elements.

\begin{python}
>>> [(x, x**2) for x in vec]
[(2, 4), (4, 16), (6, 36)]
\end{python}

Double list comprehension (all pairs).

\begin{python}
>>> vec1 = [2, 4, 6]
>>> vec2 = [4, 3, -9]
>>> [x*y for x in vec1 for y in vec2]
[8, 6, -18, 16, 12, -36, 24, 18, -54]
\end{python}

\end{frame}
%%%%%%%%%%%%%%%%%%%%%%%%%%%%%%%%%%%%%%%

%%%%%%%%%%%%%%%%%%%%%%%%%%%%%%%%%%%%%%%%%%%%%%%%%%%%%%%%%%%%%%%%%%%%%%%%%%%%%%
\section{Libraries}
%%%%%%%%%%%%%%%%%%%%%%%%%%%%%%%%%%%%%%%%%%%%%%%%%%%%%%%%%%%%%%%%%%%%%%%%%%%%%%

%%%%%%%%%%%%%%%%%%%%%%%%%%%%%%%%%%%%%%%
\begin{frame}[fragile]{Libraries}

Besides normal Python there are numerous libraries that can augment the functionality of Python.

	\spacer

Analysis:
\begin{itemize}
\item Numpy -- Numeric Python, workhorse library
\item SciPy -- Scientific Python, advanced numerical recipes
\end{itemize}

	\spacer

Data IO
\begin{itemize}
\item pyfits -- FITS file IO
\item idlsave -- load IDL \texttt{.save} files
\end{itemize}

	\spacer
\end{frame}
%%%%%%%%%%%%%%%%%%%%%%%%%%%%%%%%%%%%%%%

%%%%%%%%%%%%%%%%%%%%%%%%%%%%%%%%%%%%%%%
\begin{frame}[fragile]{Libraries cont'd}

Plotting
\begin{itemize}
\item matplotlib -- interactive plotting as in MatLab
\item GnuPlot -- interface for gnuplot
\item pyds9 -- interface to ds9 viewer
\end{itemize}

	\spacer

Miscellaneous
\begin{itemize}
\item argparse -- parse command-line arguments
\item re -- regular expressions
\end{itemize}

\end{frame}
%%%%%%%%%%%%%%%%%%%%%%%%%%%%%%%%%%%%%%%

%%%%%%%%%%%%%%%%%%%%%%%%%%%%%%%%%%%%%%%%%%%%%%%%%%%%%%%%%%%%%%%%%%%%%%%%%%%%%%
\section{Final words}
%%%%%%%%%%%%%%%%%%%%%%%%%%%%%%%%%%%%%%%%%%%%%%%%%%%%%%%%%%%%%%%%%%%%%%%%%%%%%%

%%%%%%%%%%%%%%%%%%%%%%%%%%%%%%%%%%%%%%%
\begin{frame}[fragile]{Things not covered -- Python}
	Some Python things \emph{not} covered in this presentation include:
	\spacer
	\begin{itemize}
		\item object-oriented programming
		\item unit-testing
		\item making packages/distributions
	\end{itemize}
\end{frame}
%%%%%%%%%%%%%%%%%%%%%%%%%%%%%%%%%%%%%%%

%%%%%%%%%%%%%%%%%%%%%%%%%%%%%%%%%%%%%%%
\begin{frame}[fragile]{Things not covered -- The rest}
	Other things that are not included here but equally important:
	\spacer
	\begin{itemize}
		\item version control (CVS, svn, git, bazaar, mercurial)
		\item the shell (bash, zsh)
		\item regular expressions
		\item make
		\item debugging
	\end{itemize}
\end{frame}
%%%%%%%%%%%%%%%%%%%%%%%%%%%%%%%%%%%%%%%

%%%%%%%%%%%%%%%%%%%%%%%%%%%%%%%%%%%%%%%%%%%%%%%%%%%%%%%%%%%%%%%%%%%%%%%%%%%%%%
\section*{References}
%%%%%%%%%%%%%%%%%%%%%%%%%%%%%%%%%%%%%%%%%%%%%%%%%%%%%%%%%%%%%%%%%%%%%%%%%%%%%%

%%%%%%%%%%%%%%%%%%%%%%%%%%%%%%%%%%%%%%%
\begin{frame}{Literature}
	Some of the literature used for this presentation
	\spacer
	\begin{itemize}
		\item Software carpentry -- \url{http://software-carpentry.org/}
		\item Python online documentation -- \url{http://docs.python.org/}
		\item NumPy/Scipy documentation -- \url{http://docs.scipy.org/doc/}
		\item StackOverflow -- \url{http://stackoverflow.com/questions/tagged/python}
	\end{itemize}
	
\end{frame}
%%%%%%%%%%%%%%%%%%%%%%%%%%%%%%%%%%%%%%%

%%%%%%%%%%%%%%%%%%%%%%%%%%%%%%%%%%%%%%%
\begin{frame}
	\frametitle{License}

	\begin{center}
	\includegraphics[width=1cm]{\imgpath cc.pdf}~
	\includegraphics[width=1cm]{\imgpath by.pdf}~
	\includegraphics[width=1cm]{\imgpath sa.pdf}~
	
	\spacer
	
	Copyright © \href{https://www.staff.science.uu.nl/~werkh108/}{Tim van Werkhoven}\footnote[frame]{\url{t.i.m.vanwerkhoven@gmail.com}} 2011\\
	Available online at \url{http://python101.vanwerkhoven.org}
	
	\spacer
	
	This work is licensed under a\\
	\href{http://creativecommons.org/licenses/by-sa/3.0/}{Creative Commons Attribution-ShareAlike 3.0 Unported License}.

	\end{center}
	
\end{frame}
%%%%%%%%%%%%%%%%%%%%%%%%%%%%%%%%%%%%%%%

\end{document}



