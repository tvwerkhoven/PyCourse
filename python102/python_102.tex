%% Python 102 -- Using Numpy, Scipy, matplotlib and other stuff
%%
%% Tim van Werkhoven (t.i.m.vanwerkhoven@gmail.com), 20111012
%%
%% This file is licensed under the Creative Commons Attribution-Share Alike
%% license versions 3.0 or higher, see
%% http://creativecommons.org/licenses/by-sa/3.0/

% Copyright 2004 by Till Tantau <tantau@users.sourceforge.net>.
%
% In principle, this file can be redistributed and/or modified under
% the terms of the GNU Public License, version 2.
%
% However, this file is supposed to be a template to be modified
% for your own needs. For this reason, if you use this file as a
% template and not specifically distribute it as part of a another
% package/program, I grant the extra permission to freely copy and
% modify this file as you see fit and even to delete this copyright
% notice. 

% Generate either a full presentation, or just handouts. Handouts don't 
% include the overlay things and 'compact' a presentation on less slides.
\documentclass[xetex,10pt]{beamer}
%\documentclass[xetex,10pt,handout]{beamer}
%\documentclass[xetex,mathserif,serif,10pt, handout]{beamer}

%%% Theme %%%%%%%%%%%%%%%%%%%%%%%%%%%%%%%%%%%%%%%%%%%%%%%%%%%%%%%%%%%%%%%%%%%%

\usetheme{Madrid}

%%% Packages %%%%%%%%%%%%%%%%%%%%%%%%%%%%%%%%%%%%%%%%%%%%%%%%%%%%%%%%%%%%%%%%%

% Prefer British spelling & hyphenation
\usepackage[british]{babel}
\usepackage{xunicode}
\usepackage{xltxtra}
\defaultfontfeatures{Mapping=tex-text}

\usepackage{graphicx}
\usepackage{amsmath, amsthm, amssymb}
\usepackage{hyperref}

% from http://blog.miliauskas.lt/2008/09/python-syntax-highlighting-in-latex.html
% from http://licejus.lt/~gintas/files/pythonlisting.tex
%\include{pythonlisting}
\usepackage{color}
\usepackage[procnames]{listings}
\usepackage{setspace}
\usepackage{textcomp}

% from http://www.nabble.com/zorder-seems-to-cause-problems-when-embed-python-in-latex-files-td19949285.html
% from http://www.imada.sdu.dk/~ehmsen/python.sty
%\usepackage{python}

\usepackage{tikz}
\usetikzlibrary{arrows,positioning} 

%%% Python code highlighting %%%%%%%%%%%%%%%%%%%%%%%%%%%%%%%%%%%%%%%%%%%%%%%%


\definecolor{gray}{gray}{0.5}
\definecolor{green}{rgb}{0,0.5,0}
\definecolor{lightgreen}{rgb}{0,0.7,0}
\definecolor{purple}{rgb}{0.5,0,0.5}
\definecolor{darkred}{rgb}{0.5,0,0}
\definecolor{gray}{gray}{0.5}

% This gives syntax highlighting in the python environment
\renewcommand{\lstlistlistingname}{Python Snippets}
\renewcommand{\lstlistingname}{Python Snippet}

\lstdefinelanguage{pythontim}{
basicstyle=\ttfamily\small\setstretch{1},
morestring=[s]{"""}{"""},
morestring=[s]{'''}{'''},
morestring=[s]{'}{'},
morestring=[s]{"}{"},
stringstyle=\color{red},
showstringspaces=false,
alsoletter={1234567890},
%otherkeywords={\ , \}, \{},
keywordstyle=\color{blue},
emph={access,and,as,break,class,continue,def,del,elif,else,%
except,exec,finally,for,from,global,if,import,in,is,%
lambda,not,or,pass,print,raise,return,try,while,assert,yield},
emphstyle=\color{orange}\bfseries,
emph={[2]True, False, None, self},
emphstyle=[2]\color{green},
emph={[3]from, import, as},
emphstyle=[3]\color{blue},
upquote=true,
%moredelim=**[is][\only<+>{\makebox[0cm]{•}}\hspace{1em}]{@}{@},
moredelim=**[is][\only<+>{\hspace{-2mm}\makebox[2mm]{•}}]{@}{@},
morecomment=[l]{\#},
commentstyle=\color{gray}\slshape,
literate=*{:}{{\textcolor{blue}:}}{1}%
	{=}{{\textcolor{blue}=}}{1}%
	{-}{{\textcolor{blue}-}}{1}%
	{+}{{\textcolor{blue}+}}{1}%
	{*}{{\textcolor{blue}*}}{1}%
	{!}{{\textcolor{blue}!}}{1}%
	{(}{{\textcolor{blue}(}}{1}%
	{)}{{\textcolor{blue})}}{1}%
	{[}{{\textcolor{blue}[}}{1}%
	{]}{{\textcolor{blue}]}}{1}%
	{<}{{\textcolor{blue}<}}{1}%
	{>}{{\textcolor{blue}>}}{1},%
emph={[5]ArithmeticError,AssertionError,AttributeError,BaseException,%
DeprecationWarning,EOFError,Ellipsis,EnvironmentError,Exception,%
False,FloatingPointError,FutureWarning,GeneratorExit,IOError,%
ImportError,ImportWarning,IndentationError,IndexError,KeyError,%
KeyboardInterrupt,LookupError,MemoryError,NameError,None,%
NotImplemented,NotImplementedError,OSError,OverflowError,%
PendingDeprecationWarning,ReferenceError,RuntimeError,RuntimeWarning,%
StandardError,StopIteration,SyntaxError,SyntaxWarning,SystemError,%
SystemExit,TabError,True,TypeError,UnboundLocalError,UnicodeDecodeError,%
UnicodeEncodeError,UnicodeError,UnicodeTranslateError,UnicodeWarning,%
UserWarning,ValueError,Warning,ZeroDivisionError,abs,all,any,apply,%
basestring,bool,buffer,callable,chr,classmethod,cmp,coerce,compile,%
complex,copyright,credits,delattr,dict,dir,divmod,enumerate,eval,%
execfile,exit,file,filter,float,frozenset,getattr,globals,hasattr,%
hash,help,hex,id,input,int,intern,isinstance,issubclass,iter,len,%
license,list,locals,long,map,max,min,object,oct,open,ord,pow,property,%
quit,range,raw_input,reduce,reload,repr,reversed,round,set,setattr,%
slice,sorted,staticmethod,str,sum,super,tuple,type,unichr,unicode,%
vars,xrange,zip},
emphstyle=[5]\color{purple}\bfseries,
escapechar=\~,
%xrightmargin=2mm,
framextopmargin=1mm, frame=shadowbox,
rulesepcolor=\color{blue},
}

\lstnewenvironment{python}[1][]{
\lstset{
language=pythontim,
xleftmargin=2mm,
framexleftmargin=3mm, 
#1
}}{}
\lstnewenvironment{pythonln}[1][]{
\lstset{
language=pythontim,
numbers=left,
numberstyle=\ttfamily\scriptsize,
xleftmargin=6mm,
framexleftmargin=7mm,
#1
}}{}


\def\pythoni{\lstinline[language=pythontim]}

%%% TikZ settings %%%%%%%%%%%%%%%%%%%%%%%%%%%%%%%%%%%%%%%%%%%%%%%%%%%%%%%%%%%

% For every picture that defines or uses external nodes, you'll have to
% apply the 'remember picture' style. To avoid some typing, we'll apply
% the style to all pictures.
\tikzstyle{every picture}+=[remember picture]
\tikzstyle{na} = [baseline=-.5ex]

\tikzset{
    %Define standard arrow tip
    >=stealth',
    %Define style for boxes
    punkt/.style={
           rectangle,
           rounded corners,
           draw=black, very thick,
           text width=6.5em,
           minimum height=2em,
           text centered},
    % Define arrow style
    pil/.style={
           ->,
           thick,
           shorten <=2pt,
           shorten >=2pt,}
}

%%% Custom commands  %%%%%%%%%%%%%%%%%%%%%%%%%%%%%%%%%%%%%%%%%%%%%%%%%%%%%%%%%

\def\imgpath{../00-img/}

\def\green{\color{green}}
\def\orange{\color{orange}}
\def\red{\color{red}}

\newcommand{\squash}[1]{\parbox{0pt}{#1}}

\def\spacer{\vspace*{1em}}

\newcommand{\pypypy}[1]{\footnote[frame]{Py3: #1}}

%%% Presentation layout %%%%%%%%%%%%%%%%%%%%%%%%%%%%%%%%%%%%%%%%%%%%%%%%%%%%%%

% No navigation symbols please
\setbeamertemplate{navigation symbols}{}

% Cover with some transparency
\setbeamercovered{dynamic}
%\setbeamercovered{invisible}

% Delete this, if you do not want the table of contents to pop up at
% the beginning of each section:
\AtBeginSection[]
{
 \begin{frame}<beamer>
   \frametitle{Outline}
   \tableofcontents[currentsection]
 \end{frame}
}

% If you wish to uncover everything in a step-wise fashion, uncomment
% the following command: 
%\beamerdefaultoverlayspecification{<+->}

%%% Document details %%%%%%%%%%%%%%%%%%%%%%%%%%%%%%%%%%%%%%%%%%%%%%%%%%%%%%%%%

% This is only inserted into the PDF information catalog.

\hypersetup{
pdfauthor = {Tim van Werkhoven},
pdftitle = {Python 102 using numpy},
pdfsubject = {Python 102 using numpy},
pdfkeywords = {Python, NumPy, SciPy, matplotlib},
pdfcreator = {LaTeX with hyperref package},
pdfproducer = {pdfLaTeX},
colorlinks = false,
breaklinks = true,
linkcolor = red,                    % color of internal links
citecolor = green,                  % color of links to bibliography
filecolor = magenta,                % color of file links
urlcolor = blue,                    % color of external links
}

%%% Authors etc. %%%%%%%%%%%%%%%%%%%%%%%%%%%%%%%%%%%%%%%%%%%%%%%%%%%%%%%%%%%%%

\title{Python 102}
\subtitle{Using Python for numerical data analysis}
\author[\href{http://work.vanwerkhoven.org/}{Tim van Werkhoven}]%
{\href{http://work.vanwerkhoven.org/}{Tim van Werkhoven}\\%
\url{t.i.m.vanwerkhoven@gmail.com}}
\institute[SIU, UU]{%
\href{http://www.astro.uu.nl/}{Sterrekundig Instituut Utrecht}, %
\href{http://www.uu.nl}{Utrecht University}\\[1em]
\includegraphics[height=10mm]{\imgpath logo_uu.pdf}
}
\date{October 2011}

%%% Frames %%%%%%%%%%%%%%%%%%%%%%%%%%%%%%%%%%%%%%%%%%%%%%%%%%%%%%%%%%%%%%%%%%%

% Structuring a talk is a difficult task and the following structure
% may not be suitable. Here are some rules that apply for this
% solution: 

% - Exactly two or three sections (other than the summary).
% - At *most* three subsections per section.
% - Talk about 30s to 2min per frame. So there should be between about
%   15 and 30 frames, all told.

% - A conference audience is likely to know very little of what you
%   are going to talk about. So *simplify*!
% - In a 20min talk, getting the main ideas across is hard
%   enough. Leave out details, even if it means being less precise than
%   you think necessary.
% - If you omit details that are vital to the proof/implementation,
%   just say so once. Everybody will be happy with that.

\begin{document}

%\setbeameroption{show notes on second screen} 

%%%%%%%%%%%%%%%%%%%%%%%%%%%%%%%%%%%%%%%
\begin{frame}
  \titlepage
\end{frame}
%%%%%%%%%%%%%%%%%%%%%%%%%%%%%%%%%%%%%%%

%%%%%%%%%%%%%%%%%%%%%%%%%%%%%%%%%%%%%%%%%%%%%%%%%%%%%%%%%%%%%%%%%%%%%%%%%%%%%%
\section{Introduction}
%%%%%%%%%%%%%%%%%%%%%%%%%%%%%%%%%%%%%%%%%%%%%%%%%%%%%%%%%%%%%%%%%%%%%%%%%%%%%%

%%%%%%%%%%%%%%%%%%%%%%%%%%%%%%%%%%%%%%%
\begin{frame}[fragile]
	\frametitle{This presentation}
	
	This presentation is part two of two.

	\spacer
	\begin{itemize}
		\item Before, 101: basic Python language syntax
		\begin{itemize}
			\item \emph{objects, data types, flow control, functions, slicing}
		\end{itemize}
		\item Now, 102: (numerical) data processing, plotting
		\begin{itemize}
			\item \emph{packages, NumPy, pyfits, matplotlib, fitting surface}
		\end{itemize}
		\item After that: anything can happen. Suggestions?
	\end{itemize}
	
\end{frame}
%%%%%%%%%%%%%%%%%%%%%%%%%%%%%%%%%%%%%%%

%%%%%%%%%%%%%%%%%%%%%%%%%%%%%%%%%%%%%%%
\begin{frame}{Prerequisite knowledge}
	
	This lecture assumes the following understanding of Python
	
	\spacer
	
	\begin{itemize}
		\item Running Python interactively or as script
		\item Python variables, datatypes, control flow
		\item Aliasing, slicing, data manipulation.
	\end{itemize}
	
	\spacer

	This is covered in the Python 101 lecture.

\end{frame}
%%%%%%%%%%%%%%%%%%%%%%%%%%%%%%%%%%%%%%%

%%%%%%%%%%%%%%%%%%%%%%%%%%%%%%%%%%%%%%%
\begin{frame}
    \frametitle{Outline}
    %\note{Some notes for the current slide}
    \setcounter{tocdepth}{2}
    \tableofcontents
    % You might wish to add the option [pausesections]
\end{frame}
%%%%%%%%%%%%%%%%%%%%%%%%%%%%%%%%%%%%%%%

%%%%%%%%%%%%%%%%%%%%%%%%%%%%%%%%%%%%%%%%%%%%%%%%%%%%%%%%%%%%%%%%%%%%%%%%%%%%%%
\section{Packages}
%%%%%%%%%%%%%%%%%%%%%%%%%%%%%%%%%%%%%%%%%%%%%%%%%%%%%%%%%%%%%%%%%%%%%%%%%%%%%%

%%%%%%%%%%%%%%%%%%%%%%%%%%%%%%%%%%%%%%%
\begin{frame}[fragile]{Package overview}
	Python can be extended with \emph{packages}. Some are included with Python\footnote[frame]{See \url{http://docs.python.org/library/index.html} for an overview}, others have to be installed separately.
	\spacer
	\begin{itemize}
		\item \href{http://docs.python.org/library/argparse.html}{\texttt{argparse}} (py) -- parse command-line arguments
		\item \href{http://docs.python.org/library/re.html}{\texttt{re}} (py) -- regular expressions
		\item \href{http://docs.python.org/library/sys.html}{\texttt{sys}}/%
		\href{http://docs.python.org/library/os.html}{\texttt{os}} (py) -- system tools
		\pause
		\item \texttt{scipy} -- scientific tools
		\item \texttt{numpy} -- scipy core utilities
		\pause
		\item \texttt{pyfits} -- FITS file IO
		\item \texttt{idlsave} -- load IDL \texttt{.save} files\footnote[frame]{\url{http://www.physics.wisc.edu/~craigm/idl/savefmt/}}
		\pause
		\item \texttt{matplotlib} -- interactive plotting
		\item \texttt{pyds9} -- interface to ds9
		\pause
		\item \texttt{ipython} -- enhanced python shell
	\end{itemize}
\end{frame}
%%%%%%%%%%%%%%%%%%%%%%%%%%%%%%%%%%%%%%%

%%%%%%%%%%%%%%%%%%%%%%%%%%%%%%%%%%%%%%%
\begin{frame}[fragile]{Package overview II}
	Other packages that look promising.
	\spacer
	\begin{itemize}
		\item \texttt{h5py} -- HDF5 interface
		\item \texttt{tables} -- managing hierarchical datasets
		\pause
		\item \texttt{pyfftw3} -- bindings to FFTW3
		\item \texttt{mpi4py} -- MPI for Python
		\pause
		\item \texttt{kapteyn} -- The Kapteyn Package
		\item \texttt{vo} -- parse, validate and generate VOTable XML
		\pause
		\item \texttt{pybtex} -- BibTeX-compatible bibliography processor
		\item \texttt{xlrd} -- extract data from Microsoft Excel files
		\item \texttt{sympy} -- symbolic mathematics
	\end{itemize}
\end{frame}
%%%%%%%%%%%%%%%%%%%%%%%%%%%%%%%%%%%%%%%

%%%%%%%%%%%%%%%%%%%%%%%%%%%%%%%%%%%%%%%%%%%%%%%%%%%%%%%%%%%%%%%%%%%%%%%%%%%%%%
\subsection{Obtaining packages}
%%%%%%%%%%%%%%%%%%%%%%%%%%%%%%%%%%%%%%%%%%%%%%%%%%%%%%%%%%%%%%%%%%%%%%%%%%%%%%

%%%%%%%%%%%%%%%%%%%%%%%%%%%%%%%%%%%%%%%
\begin{frame}[fragile]{Obtaining packages}
	To get a package, install it in one of several ways:
	
	\spacer
	
	\begin{itemize}
		\item Package is already installed (Enthought\footnote[frame]{See \url{http://www.enthought.com/products/epdlibraries.php}}, Activestate\footnote[frame]{See \url{http://www.activestate.com/activepython}})
		\pause
		\item Use package manager to install (\texttt{apt}, \texttt{emerge}, \texttt{fink}, \texttt{macports}, \texttt{homebrew})
		\pause
		\item Manually download and install (not recommended)
	\end{itemize}
	
	\spacer
	\pause
	
	Using a package manager is preferred because of dependencies tracking and updates.
\end{frame}
%%%%%%%%%%%%%%%%%%%%%%%%%%%%%%%%%%%%%%%

%%%%%%%%%%%%%%%%%%%%%%%%%%%%%%%%%%%%%%%
\begin{frame}[fragile]{Using packages}
	
	To use a package, run \pythoni{import}\verb! <package_name> [as <alias>]!.
	\spacer
\begin{python}
>>> import numpy as N
>>> N.arange(10)
array([0, 1, 2, 3, 4, 5, 6, 7, 8, 9])
\end{python}
	\spacer
	\pause
	You can also import specific functions to the current \emph{namespace}.
	\spacer
\begin{python}
>>> from numpy import arange
>>> arange(10)
array([0, 1, 2, 3, 4, 5, 6, 7, 8, 9])
\end{python}

\end{frame}
%%%%%%%%%%%%%%%%%%%%%%%%%%%%%%%%%%%%%%%

%%%%%%%%%%%%%%%%%%%%%%%%%%%%%%%%%%%%%%%
\begin{frame}[fragile]{Using packages -- warning}
	One can also naively run
	
	\begin{python}
	from package import *
	\end{python}

	to get all functionality into the current namespace. Generally, this is \textbf{bad}.
	\spacer
	\pause
	
	How will I know why explicit imports are better than the wild-card form?\\
	\pause
	You will know when your code you try to read six months from now.\\
	
	\spacer
	\pause

	In general:
	\begin{itemize}
	\item \pythoni{import numpy}
	\item \pythoni{import numpy as np}
	\item \pythoni{from numpy import arange}
	\end{itemize}
	
\end{frame}
%%%%%%%%%%%%%%%%%%%%%%%%%%%%%%%%%%%%%%%

%%%%%%%%%%%%%%%%%%%%%%%%%%%%%%%%%%%%%%%%%%%%%%%%%%%%%%%%%%%%%%%%%%%%%%%%%%%%%%
\section{Miscellaneous Examples}
%%%%%%%%%%%%%%%%%%%%%%%%%%%%%%%%%%%%%%%%%%%%%%%%%%%%%%%%%%%%%%%%%%%%%%%%%%%%%%

%%%%%%%%%%%%%%%%%%%%%%%%%%%%%%%%%%%%%%%
\begin{frame}[fragile]{Various examples}
	The following are some didactical Python examples from \url{http://wiki.python.org/moin/SimplePrograms}.
\end{frame}
%%%%%%%%%%%%%%%%%%%%%%%%%%%%%%%%%%%%%%%

%%%%%%%%%%%%%%%%%%%%%%%%%%%%%%%%%%%%%%%
\begin{frame}[fragile]{Example -- for \& enumerate}
	\begin{python}[caption=\texttt{py102-example1a-enumerate.py}]
my_list = ['john', 'pat', 'gary', 'michael']
for i, name in enumerate(my_list):
  print "iteration %i is %s" % (i, name)
	\end{python}
	\spacer
	\pause
	\begin{itemize}
	\item \pythoni{enumerate} returns a tuple of the count and value from a sequence.
	\pause
	\item tuple unpacking is done automatically in \pythoni{i, name in enumerate(my_list)}
	\pause
	\item string formatting goes as \pythoni{'format string' % (value1, value2, ...)}
	\footnote[frame]{\url{http://docs.python.org/library/stdtypes.html\#string-formatting-operations}}
	\end{itemize}
\end{frame}
%%%%%%%%%%%%%%%%%%%%%%%%%%%%%%%%%%%%%%%

%%%%%%%%%%%%%%%%%%%%%%%%%%%%%%%%%%%%%%%
\begin{frame}[fragile]{Example -- dictionaries, generator expressions}
	\begin{python}[caption=\texttt{py102-example1b-dictsum.py}]
prices = {'apple': 0.40, 'banana': 0.50}
my_purchase = {
  'apple': 1,
  'banana': 6}
grocery_bill = sum(prices[fruit] * my_purchase[fruit]
                 for fruit in my_purchase)
print 'I owe the grocer $%.2f' % grocery_bill	
\end{python}
	\spacer
	\pause
	\begin{itemize}
	\item A dictionary can store arbitrary key $\rightarrow$ value mappings.
	\pause
	\item \pythoni{sum()} returns the sum of an \emph{iterable} object.
	\pause
	\item \pythoni{'%.2f'} formats a value as a floating point with two digits after the decimal.
	\end{itemize}
\end{frame}
%%%%%%%%%%%%%%%%%%%%%%%%%%%%%%%%%%%%%%%

%%%%%%%%%%%%%%%%%%%%%%%%%%%%%%%%%%%%%%%
\begin{frame}[fragile]{Example -- triple-quotes, while loop }
	\begin{python}[caption=\texttt{py102-example1c-while.py}]
REFRAIN = """
%d bottles of beer on the wall,
%d bottles of beer,
take one down, pass it around,
%d bottles of beer on the wall!
"""
bottles_of_beer = 99
while bottles_of_beer > 1:
    print REFRAIN % (bottles_of_beer, bottles_of_beer,
        bottles_of_beer - 1)
    bottles_of_beer -= 1
\end{python}
	\spacer
	\pause
	\begin{itemize}
		\item Consistent variable naming, constants, functions, variables.
		\pause
		\item \pythoni{-=} and \pythoni{+=} do exactly what you think (but \emph{no} \pythoni{--} or \pythoni{++})\footnote[frame]{Overview of operators: \url{http://docs.python.org/reference/lexical_analysis.html\#operators}}
	\end{itemize}
\end{frame}
%%%%%%%%%%%%%%%%%%%%%%%%%%%%%%%%%%%%%%%

%%%%%%%%%%%%%%%%%%%%%%%%%%%%%%%%%%%%%%%
\begin{frame}[fragile]{Example -- Generators}
	\begin{python}[caption=\texttt{py102-example1d-generators.py}]
def fib():
    a, b = 0, 1
    while True:
        yield a
        a, b = b, a + b

fib_gen = fib()
[fib_gen.next() for i in xrange(20)]
\end{python}
	\spacer
	\pause
	\begin{itemize}
		\item \emph{generators} are powerful tools to create iterators
		\pause
		\item generators \pythoni{yield} a value when \pythoni{next()} is called
		\pause
		\item generators do not take up memory, as opposed to lists
		\pause
		\item \pythoni{range}\pypypy{\pythoni{xrange} replaced \pythoni{range}} is a generator for numbers
	\end{itemize}
\end{frame}
%%%%%%%%%%%%%%%%%%%%%%%%%%%%%%%%%%%%%%%

%%%%%%%%%%%%%%%%%%%%%%%%%%%%%%%%%%%%%%%%%%%%%%%%%%%%%%%%%%%%%%%%%%%%%%%%%%%%%%
\section{Fitting a surface}
%%%%%%%%%%%%%%%%%%%%%%%%%%%%%%%%%%%%%%%%%%%%%%%%%%%%%%%%%%%%%%%%%%%%%%%%%%%%%%

%%%%%%%%%%%%%%%%%%%%%%%%%%%%%%%%%%%%%%%
\begin{frame}[fragile]{Fitting a surface}
	\vfill
	\emph{Given an arbitrary surface $\phi$, fit a set of (Zernike) polynomials to it.}
	\vfill
\end{frame}
%%%%%%%%%%%%%%%%%%%%%%%%%%%%%%%%%%%%%%%

%%%%%%%%%%%%%%%%%%%%%%%%%%%%%%%%%%%%%%%%%%%%%%%%%%%%%%%%%%%%%%%%%%%%%%%%%%%%%%
\subsection{Zernike polynomials}
%%%%%%%%%%%%%%%%%%%%%%%%%%%%%%%%%%%%%%%%%%%%%%%%%%%%%%%%%%%%%%%%%%%%%%%%%%%%%%

%%%%%%%%%%%%%%%%%%%%%%%%%%%%%%%%%%%%%%%
\begin{frame}[fragile]{Zernike polynomials}
	
	Zernike polynomials are defined as:
	
	\[
	\begin{array}{lcl}
	Z^{\,m}_n(\rho,\varphi) &=& R^{\,m}_{n}(\rho)\,\cos(m\,\varphi) \! \\
	Z^{\,-\!m}_n(\rho,\varphi) &=& R^{\,m}_{n}(\rho)\,\sin(m\,\varphi), \! \\
  	\end{array}
	\]
	\pause 
	with $R^{\,m}_{n}(\rho)$ given by
	
	\[
	R^{\,m}_{n}(\rho) = \left\{
	\begin{array}{l l}
    \! \sum_{k=0}^{(n-m)/2} \!\!\! \frac{(-1)^k\,(n-k)!}{k!\,((n+m)/2-k)!\,((n-m)/2-k)!} \;\rho^{n-2\,k} & \quad \text{if $m-n$ is even}\\
    0 & \quad \text{if $m-n$ is odd}\\
  	\end{array} \right.
	\]
	\spacer

\end{frame}
%%%%%%%%%%%%%%%%%%%%%%%%%%%%%%%%%%%%%%%

%%%%%%%%%%%%%%%%%%%%%%%%%%%%%%%%%%%%%%%
\begin{frame}[fragile]{Theory}
	
	Any surface can be expressed as 
	
	\[
	\phi = \sum_i^N a_i Z_i
	\]
	
	where $a_i$ are scalar coefficients. \pause Taking the inner product with each $Z_j$ on both sides gives
	
	\[
	\begin{array}{ccc}
	\langle \phi,Z_1 \rangle &=& \sum a_i \; \langle Z_i, Z_1 \rangle\\
	\vdots &  & \vdots \\
	\langle \phi,Z_N \rangle &=& \sum a_i \; \langle Z_i, Z_N \rangle\\
  	\end{array}
	\]
	
	\pause
	or as matrix multiplication:
	
	\[
	 \begin{pmatrix}
	 \langle \phi,Z_1 \rangle \\
	 \vdots \\
	 \langle \phi,Z_N \rangle \\
	 \end{pmatrix}
	 = M_{\mathrm{cov}} 
	 \begin{pmatrix}
	 a_1 \\
	 \vdots \\
	 a_N \\
	 \end{pmatrix}
	\]

\end{frame}
%%%%%%%%%%%%%%%%%%%%%%%%%%%%%%%%%%%%%%%

%%%%%%%%%%%%%%%%%%%%%%%%%%%%%%%%%%%%%%%
\begin{frame}[fragile]{Theory}
	
	Given $\phi$ and $Z_j$, we then solve for $a_j$ using
	
	\spacer

	\[
	 M_{\mathrm{cov}}^{\,-1} 
	 \begin{pmatrix}
	 \langle \phi,Z_1 \rangle \\
	 \vdots \\
	 \langle \phi,Z_N \rangle \\
	 \end{pmatrix}
	 =
	 \begin{pmatrix}
	 a_1 \\
	 \vdots \\
	 a_N \\
	 \end{pmatrix}
	\]

\end{frame}
%%%%%%%%%%%%%%%%%%%%%%%%%%%%%%%%%%%%%%%

%%%%%%%%%%%%%%%%%%%%%%%%%%%%%%%%%%%%%%%%%%%%%%%%%%%%%%%%%%%%%%%%%%%%%%%%%%%%%%
\subsection{Implementation in Python}
%%%%%%%%%%%%%%%%%%%%%%%%%%%%%%%%%%%%%%%%%%%%%%%%%%%%%%%%%%%%%%%%%%%%%%%%%%%%%%

%%%%%%%%%%%%%%%%%%%%%%%%%%%%%%%%%%%%%%%
\begin{frame}[fragile]{Analysis in Python}

Implementing this in Python step by step. See results in \texttt{py102-example2-zernike.py}.

\pause
\spacer

Tell interpreter to use \verb!python2.7!, specify encoding, document file.
\begin{pythonln}[firstnumber=1]
#!/usr/bin/env python2.7
# -*- coding: utf-8 -*-

"""
Description on what this file does goes here
"""
\end{pythonln}
\end{frame}
%%%%%%%%%%%%%%%%%%%%%%%%%%%%%%%%%%%%%%%

%%%%%%%%%%%%%%%%%%%%%%%%%%%%%%%%%%%%%%%
\begin{frame}[fragile]{Import packages}

We need extra libraries to do the hard work, import them before usage

\spacer

\begin{pythonln}[firstnumber=21]
import pyfits
import numpy as N
from scipy.misc import factorial as fac
\end{pythonln}
\spacer

Three different import methods: regular, as alias or specific functions.
\end{frame}
%%%%%%%%%%%%%%%%%%%%%%%%%%%%%%%%%%%%%%%

%%%%%%%%%%%%%%%%%%%%%%%%%%%%%%%%%%%%%%%
\begin{frame}[fragile]{Coordinate grids}

Before we can construct Zernike polynomials, we need a coordinate grid.

\spacer
\begin{pythonln}[firstnumber=87]
@@grid = (N.indices((256, 256), dtype=N.float) - 128) / 128.0
@@grid_rho = (grid**2.0).sum(0)**0.5
@@grid_phi = N.arctan2(grid[0], grid[1])
@@grid_mask = grid_rho <= 1
\end{pythonln}
\end{frame}
%%%%%%%%%%%%%%%%%%%%%%%%%%%%%%%%%%%%%%%

%%%%%%%%%%%%%%%%%%%%%%%%%%%%%%%%%%%%%%%
\begin{frame}[fragile]{Zernikes in Python}
Implement Zernike polynomials in Python
\spacer
	\[
	\begin{array}{lcl}
	Z^{\,m}_n(\rho,\varphi) &=& R^{\,m}_{n}(\rho)\,\cos(m\,\varphi) \! \\
	Z^{\,-\!m}_n(\rho,\varphi) &=& R^{\,m}_{n}(\rho)\,\sin(m\,\varphi), \! \\
  	\end{array}
	\]
\spacer
\begin{pythonln}[firstnumber=49]
def zernike(m, n, rho, phi):
  """
  Calculate Zernike polynomial (m, n) given a grid of radial
  coordinates rho and azimuthal coordinates phi.
  """
@@  if (m > 0): return zernike_rad(m, n, rho) * N.cos(m * phi)
@@  if (m < 0): return zernike_rad(-m, n, rho) * N.sin(-m * phi)
@@  return zernike_rad(0, n, rho)
\end{pythonln}
\end{frame}
%%%%%%%%%%%%%%%%%%%%%%%%%%%%%%%%%%%%%%%

%%%%%%%%%%%%%%%%%%%%%%%%%%%%%%%%%%%%%%%
\begin{frame}[fragile]{Zernikes in Python}
	\[
	\sum_{k=0}^{(n-m)/2} \!\!\! \frac{(-1)^k\,(n-k)!}{k!\,((n+m)/2-k)!\,((n-m)/2-k)!} \;\rho^{n-2\,k}
	\]
\begin{pythonln}[firstnumber=26]
@@def zernike_rad(m, n, rho):
@  """
  Calculate the radial component of Zernike polynomial (m, n)
  given a grid of radial coordinates rho.
  """@
@@  wf = N.zeros(rho.shape)
@@  if (n-m % 2):
    return wf
  
@@  for k in xrange((n-m)/2+1):
@@    wf += rho**(n-2.0*k) * (-1.0)**k * fac(n-k) / \
    	( fac(k) * fac( (n+m)/2.0 - k ) * \
      fac( (n-m)/2.0 - k ) )
  
@@  return rho
\end{pythonln}
\end{frame}
%%%%%%%%%%%%%%%%%%%%%%%%%%%%%%%%%%%%%%%

%%%%%%%%%%%%%%%%%%%%%%%%%%%%%%%%%%%%%%%
\begin{frame}[fragile]{Generate test surface}

Compute a list of 25 Zernikes (\pythoni{zernikel()} uses single-indexing for Zernikes)
\spacer
\begin{pythonln}[firstnumber=94]
zern_list = [zernikel(i, grid_rho, grid_phi) * grid_mask \
  for i in xrange(25)]
\end{pythonln}

\spacer
\pause
Generate test surface

\begin{pythonln}[firstnumber=98]
test_vec = N.random.random(25)
test_surf = sum(val * zernikel(i, grid_rho, grid_phi) \
  for (i, val) in enumerate(test_vec))
\end{pythonln}
\end{frame}
%%%%%%%%%%%%%%%%%%%%%%%%%%%%%%%%%%%%%%%

%%%%%%%%%%%%%%%%%%%%%%%%%%%%%%%%%%%%%%%
\begin{frame}[fragile]{Compute covariance matrix}

Calculate covariance between all Zernike polynomials

\spacer

\begin{pythonln}[firstnumber=105]
cov_mat = N.array([[N.sum(zerni * zernj) \
  for zerni in zern_list] \
    for zernj in zern_list])
\end{pythonln}

\spacer
\pause
Invert covariance matrix using singular value decomposition (pseudo-inverse)

\spacer

\begin{pythonln}[firstnumber=108]
cov_mat_in = N.linalg.pinv(cov_mat)
\end{pythonln}

\spacer
\pause
Calculate the inner product of each Zernike mode with the test surface

\spacer

\begin{pythonln}[firstnumber=112]
wf_zern_inprod = N.array([N.sum(test_surf * zerni) \
  for zerni in zern_list])
\end{pythonln}


\end{frame}
%%%%%%%%%%%%%%%%%%%%%%%%%%%%%%%%%%%%%%%

%%%%%%%%%%%%%%%%%%%%%%%%%%%%%%%%%%%%%%%
\begin{frame}[fragile]{Solve equation}

Given the inner-product vector and the inverse of the covariance matrix, solve for the vector of coefficients.

\begin{pythonln}[firstnumber=117]
rec_wf_pow = N.dot(cov_mat_in, wf_zern_inprod)
\end{pythonln}

\spacer
\pause
Reconstruct the surface
\begin{pythonln}[firstnumber=117]
rec_wf = sum(val * zernikel(i, grid_rho, grid_phi) \
  for (i, val) in enumerate(rec_wf_pow))
\end{pythonln}

\spacer
\pause
Check solution
\begin{pythonln}[firstnumber=123]
print test_vec-rec_wf_pow
print N.allclose(test_vec, rec_wf_pow)
print N.allclose(test_surf, rec_wf)
\end{pythonln}

\end{frame}
%%%%%%%%%%%%%%%%%%%%%%%%%%%%%%%%%%%%%%%

%%%%%%%%%%%%%%%%%%%%%%%%%%%%%%%%%%%%%%%%%%%%%%%%%%%%%%%%%%%%%%%%%%%%%%%%%%%%%%
\subsection{Plotting results}
%%%%%%%%%%%%%%%%%%%%%%%%%%%%%%%%%%%%%%%%%%%%%%%%%%%%%%%%%%%%%%%%%%%%%%%%%%%%%%

%%%%%%%%%%%%%%%%%%%%%%%%%%%%%%%%%%%%%%%
\begin{frame}[fragile]{2D plotting}

Use package \pythoni{pylab} for plotting.

\begin{python}
pylab.plot(xarr)
\end{python}

\spacer
\pause
More elaborate example

\begin{pythonln}[firstnumber=131]
@@fig = plt.figure(1)
@@ax = fig.add_subplot(111)

@@ax.set_xlabel('Zernike mode')
ax.set_ylabel('Power [AU]')
ax.set_title('Reconstruction quality')

@@ax.plot(test_vec, 'r-', label='Input')
ax.plot(rec_wf_pow, 'b--', label='Recovered')
@@ax.legend()
@@fig.show()
@@fig.savefig('py102-example2-plot1.pdf')
\end{pythonln}

\end{frame}
%%%%%%%%%%%%%%%%%%%%%%%%%%%%%%%%%%%%%%%

%%%%%%%%%%%%%%%%%%%%%%%%%%%%%%%%%%%%%%%
\begin{frame}[fragile]{3D plotting}

Surface plotting goes with \pythoni{imshow}.

\begin{python}
pylab.imshow(surf)
\end{python}

\spacer
\pause
More options

\begin{pythonln}[firstnumber=144]
@@fig = plt.figure(2)
@@fig.clf()

ax = fig.add_subplot(111)
@@surf_pl = ax.imshow(test_surf*grid_mask, \
  interpolation='nearest')
@@fig.colorbar(surf_pl)

fig.show()
fig.savefig('py102-example2-plot2.pdf')
fig.savefig('py102-example2-plot2.png')
\end{pythonln}

\end{frame}
%%%%%%%%%%%%%%%%%%%%%%%%%%%%%%%%%%%%%%%

%%%%%%%%%%%%%%%%%%%%%%%%%%%%%%%%%%%%%%%
\begin{frame}[fragile]{Archive results}

Storing results as \texttt{FITS} files with \pythoni{pyfits}.

\begin{python}
pyfits.writeto(fname, data)
\end{python}

\spacer
\pause
More options. Generate a header

\begin{pythonln}[firstnumber=159]
@@from time import time, asctime, \
  gmtime, localtime

@@clist = pyfits.CardList()
@@clist.append(pyfits.Card('Program', \
  'py102-example2-zernike.py') )
clist.append(pyfits.Card('Epoch', time()) )
clist.append(pyfits.Card('utctime', \
  asctime(gmtime(time()))) )
clist.append(pyfits.Card('loctime', \
  asctime(localtime(time()))) )
\end{pythonln}

\end{frame}
%%%%%%%%%%%%%%%%%%%%%%%%%%%%%%%%%%%%%%%

%%%%%%%%%%%%%%%%%%%%%%%%%%%%%%%%%%%%%%%
\begin{frame}[fragile]{Archive results}

Store two data matrices including some metadata

\spacer

\begin{pythonln}[firstnumber=165]
@@hdr = pyfits.Header(cards=clist+\
  [pyfits.Card('Desc', 'Surface input')])
@@pyfits.writeto('file1.fits', \
  test_surf, header=hdr, clobber=True, checksum=True)

hdr = pyfits.Header(cards=clist+\
  [pyfits.Card('Desc', 'Reconstructed surface')])
pyfits.writeto('file2.fits', \
  rec_wf, header=hdr, clobber=True, checksum=True)
\end{pythonln}

\end{frame}
%%%%%%%%%%%%%%%%%%%%%%%%%%%%%%%%%%%%%%%


%%%%%%%%%%%%%%%%%%%%%%%%%%%%%%%%%%%%%%%%%%%%%%%%%%%%%%%%%%%%%%%%%%%%%%%%%%%%%%
\section{Interactive Python}
%%%%%%%%%%%%%%%%%%%%%%%%%%%%%%%%%%%%%%%%%%%%%%%%%%%%%%%%%%%%%%%%%%%%%%%%%%%%%%

%%%%%%%%%%%%%%%%%%%%%%%%%%%%%%%%%%%%%%%
\begin{frame}[fragile]{Interactive python}

Interactive example... \pause \texttt{bleep bleep}

\spacer
\pause

\begin{itemize}
	\item ipython shell commands (ls, cd)
	\item data introspection/printing
	\item command/variable auto completion
	\item command history
	\item inline documentation (system and user)
	\item \ldots requests?
\end{itemize}

\end{frame}
%%%%%%%%%%%%%%%%%%%%%%%%%%%%%%%%%%%%%%%

%%%%%%%%%%%%%%%%%%%%%%%%%%%%%%%%%%%%%%%%%%%%%%%%%%%%%%%%%%%%%%%%%%%%%%%%%%%%%%
\section{Final words}
%%%%%%%%%%%%%%%%%%%%%%%%%%%%%%%%%%%%%%%%%%%%%%%%%%%%%%%%%%%%%%%%%%%%%%%%%%%%%%

%%%%%%%%%%%%%%%%%%%%%%%%%%%%%%%%%%%%%%%
\begin{frame}[fragile]{Things not covered -- The rest}
	Other things that are not included here but equally important:
	\spacer
	\begin{itemize}
		\item version control (CVS, svn, git, bazaar, mercurial)
		\item the shell (bash, zsh)
		\item regular expressions
		\item make
		\item debugging
	\end{itemize}
\end{frame}
%%%%%%%%%%%%%%%%%%%%%%%%%%%%%%%%%%%%%%%

%%%%%%%%%%%%%%%%%%%%%%%%%%%%%%%%%%%%%%%
\begin{frame}[fragile]{Learn more}
	If you want to learn more, here are some suggestions
	\spacer
	\begin{itemize}
		\item \url{http://wiki.python.org/moin/SimplePrograms}
		\item \url{http://www.secnetix.de/olli/Python/}
		\item \url{https://secure.wikimedia.org/wikibooks/en/wiki/Python_Programming}
	\end{itemize}
	
\end{frame}
%%%%%%%%%%%%%%%%%%%%%%%%%%%%%%%%%%%%%%%

%%%%%%%%%%%%%%%%%%%%%%%%%%%%%%%%%%%%%%%%%%%%%%%%%%%%%%%%%%%%%%%%%%%%%%%%%%%%%%
\section*{References}
%%%%%%%%%%%%%%%%%%%%%%%%%%%%%%%%%%%%%%%%%%%%%%%%%%%%%%%%%%%%%%%%%%%%%%%%%%%%%%

%%%%%%%%%%%%%%%%%%%%%%%%%%%%%%%%%%%%%%%
\begin{frame}{Literature}
	Some of the literature used for this presentation
	\spacer
	\begin{itemize}
		\item Software carpentry -- \url{http://software-carpentry.org/}
		\item Python online documentation -- \url{http://docs.python.org/}
		\pause
		\item NumPy/Scipy documentation -- \url{http://docs.scipy.org/doc/}
		\item matplotlib -- \url{http://matplotlib.sourceforge.net/gallery.html}
		\pause
		\item StackOverflow -- \url{http://stackoverflow.com/questions/tagged/python}
	\end{itemize}
	
\end{frame}
%%%%%%%%%%%%%%%%%%%%%%%%%%%%%%%%%%%%%%%

%%%%%%%%%%%%%%%%%%%%%%%%%%%%%%%%%%%%%%%
\begin{frame}
	\frametitle{License}

	\begin{center}
	\includegraphics[width=1cm]{\imgpath cc.pdf}~
	\includegraphics[width=1cm]{\imgpath by.pdf}~
	\includegraphics[width=1cm]{\imgpath sa.pdf}~
	
	\spacer

	Copyright © \href{http://work.vanwerkhoven.org/}{Tim van Werkhoven}\footnote[frame]{\url{t.i.m.vanwerkhoven@gmail.com}} 2011\\
	Available online at \url{http://python101.vanwerkhoven.org}
	
	\spacer

	This work is licensed under a\\
	\href{http://creativecommons.org/licenses/by-sa/3.0/}{Creative Commons Attribution-ShareAlike 3.0 Unported License}.

	\end{center}
	
\end{frame}
%%%%%%%%%%%%%%%%%%%%%%%%%%%%%%%%%%%%%%%

\end{document}



