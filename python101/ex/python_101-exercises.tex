%!TEX TS-program = xelatex
%!TEX encoding = UTF-8 Unicode

% File: python_101-exercises.tex -- exercises for Python 101 lecture
% Author: Tim van Werkhoven (t.i.m.vanwerkhoven@gmail.com, 2011)
% URL: http://python101.vanwerkhoven.org
%
% This work is licensed under the Creative Commons Attribution-NonCommercial-
% ShareAlike 3.0 Unported License. To view a copy of this license, visit 
% http://creativecommons.org/licenses/by-nc-sa/3.0/ or send a letter to 
% Creative Commons, 444 Castro Street, Suite 900, Mountain View, California, 
% 94041, USA.

%\documentclass{article}
\documentclass[draft=false]{article}

\newcommand{\answer}[1]{#1}%
%\newcommand{\answer}[1]{}%

%%% Load packages %%%%%%%%%%%%%%%%%%%%%%%%%%%%%%%%%%%%%%%%%%%%%%%%

% For clickable URLS/references/etc. within the PDF document
\usepackage[xetex, colorlinks, breaklinks]{hyperref}
% Some improvements for xetex and latex. Loads fixltx2e, metalogo, xunicode, fontspec.
% http://mirror.math.ku.edu/tex-archive/macros/latex/contrib/fontspec/fontspec.pdf
\usepackage[no-math]{fontspec}
\usepackage{xunicode}
%\usepackage{xltxtra}
% Load colors with names
%\usepackage{color}
\usepackage{xcolor}
%\usepackage{xecolour}
% For headers 
\usepackage{fancyhdr}
% Setup page size specifically
%\usepackage[twoside,paperheight=240mm,paperwidth=170mm,centering,height=186mm,width=120mm,outermargin=45mm]{geometry}
%\usepackage[xetex,twoside,paperheight=240mm,paperwidth=170mm,centering]{geometry}
\usepackage[xetex,twoside]{geometry}
\usepackage{multicol}
% for source code stuff
\usepackage[procnames]{listings}
\usepackage{setspace}
\usepackage{textcomp}

% shows various LaTeX layout settings http://ftp.snt.utwente.nl/pub/software/tex/macros/latex/contrib/layouts/
%\usepackage{layouts}
%\usepackage{layout}
% supersedes babel, http://www.ctan.org/tex-archive/macros/xetex/latex/polyglossia
\usepackage{polyglossia} 
% tiny package useful for spaces after commands
\usepackage{xspace}
% change font style of section headers http://www.ctan.org/tex-archive/macros/latex/contrib/sectsty/
\usepackage{sectsty}
% fancy verbatim http://ctan.org/tex-archive/macros/latex/contrib/fancyvrb
\usepackage{fancyvrb}
% footnotes stuff http://www.ctan.org/tex-archive/macros/latex/contrib/footmisc
%\usepackage[side]{footmisc}
\usepackage{footmisc}
% Verbose referencing
\usepackage{varioref}
% Fix link target location
\usepackage[all]{hypcap}
% Subfigures
\usepackage{subfig}
% Graphics
\usepackage[xetex]{graphicx}

%%% Hyperref setup %%%%%%%%%%%%%%%%%%%%%%%%%%%%%%%%%%%%%%%%%%%%%
\hypersetup{
pdfauthor = {Tim van Werkhoven},
pdftitle = {Python 101 exercises},
pdfsubject = {Python 101 exercises},
pdfkeywords = {Python, slicing, aliases, objects, variables},
pdfcreator = {LaTeX with hyperref package},
pdfproducer = {pdfLaTeX},
colorlinks = false,
breaklinks = true,
linkcolor = red,                    % color of internal links
citecolor = green,                  % color of links to bibliography
filecolor = magenta,                % color of file links
urlcolor = blue,                    % color of external links
}

%%% Set Font options %%%%%%%%%%%%%%%%%%%%%%%%%%%%%%%%%%%%%%%%%%%%%

\defaultfontfeatures{Mapping=tex-text, Fractions=Diagonal}

% Choose default fonts
\setmainfont[Numbers={OldStyle}]{Garamond Premier Pro}
\setsansfont[Scale=MatchLowercase]{Helvetica Neue}
\setmonofont[Scale=0.6]{Monaco}

% Math fonts aren't easy. Fontspec can only do so much (see http://www.ctan.org/tex-archive/macros/xetex/latex/fontspec). Another option is loading specific packages for math fonts, (see ftp://tug.ctan.org/pub/tex-archive/info/Free_Math_Font_Survey/survey.html and http://www.tug.org/pipermail/xetex/2008-March/009073.html) or sfmath for sans serif fonts (see http://dtrx.de/od/tex/sfmath.html).

% Using fontspec:
%\setmathrm{Arial}
%\setmathsf{Arial}
%\setmathtt{Arial}
%\setboldmathrm[BoldFont={Optima ExtraBlack}]{Optima Bold}
% With sfmath:
% \usepackage[cm]{sfmath}
% With specific packages
%\usepackage[garamond]{mathdesign}
\usepackage{eulervm}
%\usepackage[math]{kurier}
%\usepackage{cmbright}

% A somewhat satisfactory solution seems to be using unicode-math in combination with math-compatible open type fonts such as the STIX fonts (see http://www.charlietanksley.net/philtex/the-unicode-math-package-for-xelatex-and-the-stix-fonts/ and http://tex.stackexchange.com/questions/720/is-it-already-possible-to-use-the-stix-fonts and http://typophile.com/node/71171)

%\usepackage{unicode-math}
%\setmainfont{XITS}
%\setmathfont{XITS Math}
%\usepackage[math-style=ISO]{unicode-math}
% Problem: \setmathfont{} gives a bug when used in combination with amsmath...
%\setmathfont{STIXGeneral}
%\setmathfont[range=\mathit/{latin,Latin}]{Adobe Garamond Pro}
%\setmathfont[range=\mathit/{greek,Greek}]{Linux Libertine O}
%\usepackage{amsmath}

%\setmathfont{Asana-Math.otf}
%\setmathfont[range=\mathit/{greek,Greek,latin,Latin}]{Adobe Garamond Pro}

% This is how you can define custom temporary font styles

% Set default languages
\setdefaultlanguage{english}

%%% Page layout %%%%%%%%%%%%%%%%%%%%%%%%%%%%%%%%%%%%%%%%%%%%%%%

\sloppy								% don't care too much about a lot of 
									% space in sentences
\setlength{\parindent}{0ex}
\setlength{\parskip}{1ex plus 0.5ex minus 0.5ex}

%% Penalize widow and orphan lines
\widowpenalty=300
\clubpenalty=300

%\geometry{includeheadfoot,includemp,papersize={170mm,240mm},total={124mm,185mm}, outermargin=5mm, innermargin=10mm}
\geometry{paper=a4paper, layout=a4paper}
%\setlength{\hoffset}{}
%\addtolength{\hoffset}{0.5in}
%\addtolength{\textwidth}{-1in}
%\setlength{\marginparwidth}{200pt}
%\setlength{\marginparwidth}{3cm}

% Some path-shortcuts
\newcommand{\imgpath}{../../00-img/}

%%% Python code highlighting %%%%%%%%%%%%%%%%%%%%%%%%%%%%%%%%%%%%%%%%%%%%%%%%

\renewcommand{\lstlistlistingname}{Python Listings}
\renewcommand{\lstlistingname}{Code Listing}

%\newcommand{\mat}[1]{\ensuremath{\mathbf{#1}}}%
%\definecolor{gray}{gray}{0.5}
%\definecolor{green}{rgb}{0,0.5,0}
%\definecolor{lightgreen}{rgb}{0,0.7,0}
%\definecolor{purple}{rgb}{0.5,0,0.5}
%\definecolor{darkred}{rgb}{0.5,0,0}
%\definecolor{gray}{gray}{0.5}
%
%\def\green{\color{green}}
%\def\orange{\color{orange}}
%\def\red{\color{red}}

% This gives syntax highlighting in the python environment
\renewcommand{\lstlistlistingname}{Code Listings}
\renewcommand{\lstlistingname}{Code Listing}

\lstdefinelanguage{pythontim}{
basicstyle=\ttfamily\small,%\setstretch{1},
morestring=[s]{"""}{"""},
morestring=[s]{'''}{'''},
morestring=[s]{'}{'},
morestring=[s]{"}{"},
stringstyle=\color{red},
showstringspaces=false,
alsoletter={1234567890},
%otherkeywords={\ , \}, \{},
keywordstyle=\color{blue},
emph={access,and,as,break,class,continue,def,del,elif,else,%
except,exec,finally,for,from,global,if,import,in,is,%
lambda,not,or,pass,print,raise,return,try,while,assert,yield},
emphstyle=\color{orange}\bfseries,
emph={[2]True, False, None, self},
emphstyle=[2]\color{green},
emph={[3]from, import, as},
emphstyle=[3]\color{blue},
upquote=true,
%moredelim=**[is][\only<+>{\makebox[0cm]{•}}\hspace{1em}]{@}{@},
moredelim=**[is][\only<+>{\hspace{-2mm}\makebox[2mm]{•}}]{@}{@},
morecomment=[l]{\#},
commentstyle=\color{gray}\slshape,
literate=*{:}{{\textcolor{blue}:}}{1}%
	{=}{{\textcolor{blue}=}}{1}%
	{-}{{\textcolor{blue}-}}{1}%
	{+}{{\textcolor{blue}+}}{1}%
	{*}{{\textcolor{blue}*}}{1}%
	{!}{{\textcolor{blue}!}}{1}%
	{(}{{\textcolor{blue}(}}{1}%
	{)}{{\textcolor{blue})}}{1}%
	{[}{{\textcolor{blue}[}}{1}%
	{]}{{\textcolor{blue}]}}{1}%
	{<}{{\textcolor{blue}<}}{1}%
	{>}{{\textcolor{blue}>}}{1},%
emph={[5]ArithmeticError,AssertionError,AttributeError,BaseException,%
DeprecationWarning,EOFError,Ellipsis,EnvironmentError,Exception,%
False,FloatingPointError,FutureWarning,GeneratorExit,IOError,%
ImportError,ImportWarning,IndentationError,IndexError,KeyError,%
KeyboardInterrupt,LookupError,MemoryError,NameError,None,%
NotImplemented,NotImplementedError,OSError,OverflowError,%
PendingDeprecationWarning,ReferenceError,RuntimeError,RuntimeWarning,%
StandardError,StopIteration,SyntaxError,SyntaxWarning,SystemError,%
SystemExit,TabError,True,TypeError,UnboundLocalError,UnicodeDecodeError,%
UnicodeEncodeError,UnicodeError,UnicodeTranslateError,UnicodeWarning,%
UserWarning,ValueError,Warning,ZeroDivisionError,abs,all,any,apply,%
basestring,bool,buffer,callable,chr,classmethod,cmp,coerce,compile,%
complex,copyright,credits,delattr,dict,dir,divmod,enumerate,eval,%
execfile,exit,file,filter,float,frozenset,getattr,globals,hasattr,%
hash,help,hex,id,input,int,intern,isinstance,issubclass,iter,len,%
license,list,locals,long,map,max,min,object,oct,open,ord,pow,property,%
quit,range,raw_input,reduce,reload,repr,reversed,round,set,setattr,%
slice,sorted,staticmethod,str,sum,super,tuple,type,unichr,unicode,%
vars,xrange,zip},
emphstyle=[5]\color{purple}\bfseries,
escapechar=\~,
xleftmargin=2mm,
%xrightmargin=2mm,
framexleftmargin=3mm, framextopmargin=1mm, frame=shadowbox,
rulesepcolor=\color{blue},
}

\lstnewenvironment{python}[1][]{
\lstset{
language=pythontim,#1
}}{}

\def\pythoni{\lstinline[language=pythontim]}

%%% Custom commands %%%%%%%%%%%%%%%%%%%%%%%%%%%%%%%%%%%%%%%%%%%%%%%

% Some command abbreviations
\newcommand{\tsup}[1]{\textsuperscript{#1}}
\newcommand{\tsub}[1]{\textsubscript{#1}}

\renewcommand{\vec}[1]{\ensuremath{\boldsymbol{#1}}}%
\newcommand{\mat}[1]{\ensuremath{\mathbf{#1}}}%

%%% Begin document %%%%%%%%%%%%%%%%%%%%%%%%%%%%%%%%%%%%%%%%%%%%%%%

\begin{document}
\thispagestyle{empty}

%\sectionfont{\sectionrule{3ex}{3pt}{-1ex}{1pt}}

\section*{Introduction}
%%%%%%%%%%%%%%%%%%%%%%%%%%%%%%%%%%%%%%%%%%%%%%%%%%%%%%%%%%%%%%%%%%

These exercises are part of the Python 101 lecture.

\section{Reading Python}
%%%%%%%%%%%%%%%%%%%%%%%%%%%%%%%%%%%%%%%%%%%%%%%%%%%%%%%%%%%%%%%%%%

Figure out what the output of these python snippets will be, and understand why.

\setlength{\columnsep}{3em}

\subsection{Datatypes}
%%%%%%%%%%%%%%%%%%%%%%%%%%%%%%%%%%%%%%%%%%%%%%%%%%%%%%%%%%%%%%%%%%

These exercises deal with different datatype handling in Python.

\begin{multicols}{2}

\subsubsection{Adding and multiplying}

\begin{python}
1 + 1
1 * 2
"1" + "1"
"1" * 2
1, + 1,
(1,) * 2
[1,2] + [1,2]
[1,2] * 2
(1+2j) * 2
\end{python}
\answer{Adding and multiplying integers and complex numbers goes as expected. With sequences (strings, tuples and lists), multiplication with an integer means \emph{repetition}, and addition means \emph{concatenation}. \pythoni{1,} is a tuple with only one element.}
\subsubsection{(Im)mutable}

\begin{python}
a = [1, 2, 3, 4]
a[0] = 5
b = (1, 2, 3, 4)
b[0] = 5
c = ([1], 2, 3, 4)
c[0][0] = 5
\end{python}
\answer{With variable \pythoni{a}, we change element 1 (at index 0) of a mutable sequence, a list. With var \pythoni{b}, we attempt to change the first element of an immutable tuple, which raises an error. In the final example, the tuple is immutable, but the first element of the tuple is a mutable list, so we can change the value in the list. Note that \pythoni{c[0] = [5]} would have been invalid.}

\columnbreak

\subsubsection{String concatenation}
\begin{python}
print 'o' 'n' "e"
print "one" \
"two" \
"three"
print """
"hello"
'world'
"""
\end{python}

\answer{Simple string concatenation as in the first example. Strings are delimited by one or three times a single (\pythoni{'}) or double (\pythoni{"}) quote. A single backslash continues a line. Triple-quoted strings continue over multiple lines and can use quotes inside of them.}

\subsubsection{Tuple (un)packing)}

\begin{python}
a=1; b=2

temp = a
a = b
b = temp
print a, b

b, a = a, b
print a, b
\end{python}

\answer{Using tuples, one can switch variable `in place' without the need for a third variable as in other programming languages. The reason this works is that the tuple is \emph{packed} on the right side of the equation first, then transferred to the left side and \emph{unpacked} into the two variables \pythoni{a} and \pythoni{b}.}

\end{multicols}

\pagebreak

\subsection{Sequences}
%%%%%%%%%%%%%%%%%%%%%%%%%%%%%%%%%%%%%%%%%%%%%%%%%%%%%%%%%%%%%%%%%%

Here we investigate the behaviour and use of sequences in Python.

\begin{multicols}{2}

\subsubsection{Looping over sequences}

\begin{python}
a = range(1, 10)
out = 1
for i in a:
	out *= i
print out
\end{python}

\answer{\pythoni{range([start,] stop[, step])} generates a list from \pythoni{start} to \pythoni{stop} with \pythoni{step} increments. \pythoni{for} loops over all these values and multiplies \pythoni{out} with the current value all the time. The output will be 9!, or 362880.}

\begin{python}
collection = (5, 17, 4, ['mac', 'alex', 'sally'], 888, b)
for el in collection:
	print el
\end{python}

\answer{\pythoni{for} loops over all elements in something that is \emph{iterable}, such as this tuple \pythoni{collection}. Because the fourth element of \pythoni{collection} is a list, python will print the list as a whole and not the individual elements separately.}

\columnbreak

\subsubsection{Strings as sequences}

\begin{python}
mytext = "Fruit flies like a banana, time flies like 
	an arrow"
for i in mytext:
	print i

for i in mytext.split():
	print i

prev = 0
for i in map(len, mytext.split()):
	print mytext[prev:prev+i]
	prev += i+1
\end{python}

\answer{A python string is also a sequence, such that \pythoni{for} will loop over all elements. \pythoni{str.split()} splits a string at each space into a list, such that \pythoni{for} will loop over all words. The final \pythoni{for} loop gets a list of the word lengths as input, and the string slicing then prints each word. \pythoni{prev} is necessary to remember where the last word ended.}

\end{multicols}

\subsection{Functional programming}
%%%%%%%%%%%%%%%%%%%%%%%%%%%%%%%%%%%%%%%%%%%%%%%%%%%%%%%%%%%%%%%%%%

Some more interesting use of Python with \pythoni{reduce} and \emph{list comprehension}.

\begin{python}
mylist = [[[(x, y, z) for x in xrange(3)] for y in xrange(5)] for z in xrange(2)]
mylist[1][3][2]
\end{python}

\answer{Dissect this statement from outside to inside. We construct a list with \pythoni{for z in xrange(2)}. Each element of this list is a double list comprehension. The outer one is governed by \pythoni{for y in xrange(5)}, yielding a list of 5 elements. Each element of \emph{this} list is a list comprehension \pythoni{for x in xrange(3)} where the output is finally given, which is a tuple of \pythoni{(x, y, z)}. The final output is therefore a triple-nested listed with tuples as elements.}

\begin{python}
mytext = "Fruit flies like a banana, time flies like an arrow"
reduce(lambda x, y: x+" "+y, mytext.split())
\end{python}

\answer{\pythoni{reduce(func, list)} maps func on each element of list, summing along the way. \pythoni{reduce(add, [1,2,3,4])} is equivalent to \pythoni{add(add(add(1,2),3),4)}. \pythoni{str.split()} works as above. The output of this \pythoni{reduce()} statement is therefore the same as \pythoni{mytext} itself.}

\begin{python}
arglist = ['apples', 1, 'cabbage', 9, 'pears', 2, 'oranges', 5, 'ananas', 7.5, 'pineapple', 3, 'potatoes', 4 ]
ignore = ['cabbage', 'potatoes']
remlist = [i for i,el in enumerate(arglist) if el in ignore]
for idx in reversed(remlist):
	arglist[idx:idx+2] = ''
\end{python}

\answer{\pythoni{enumerate(list)} returns a tuple (index, element) for each element in \pythoni{list}. The \pythoni{if} statement in list comprehension controls the execution of the expression for each element in the \pythoni{list}. This means for all elements of \pythoni{arglist} that are in the list \pythoni{ignore}, we get the index back. \pythoni{reversed(list)} returns an iterator to the reversed version of \pythoni{list}, which for this application is the same as reversing \pythoni{list} itself. Looping over all elements in \pythoni{remlist} (which are list indexes), we set the element at that index \emph{and} the next to an empty string, effectively deleting is. The final result is that \pythoni{arglist} will no longer contain the elements cabbage, 9, potatoes and 4.}

%%%%%%%%%%%%%%%%%%%%%%%%%%%%%%%%%%%%%%%%%%%%%%%%%%%%%%%%%%%%%%%%%%

\section*{References}

%\begin{itemize}
%\end{itemize}

This work is licensed under the Creative Commons Attribution-NonCommercial-ShareAlike 3.0 Unported License.

\end{document}
